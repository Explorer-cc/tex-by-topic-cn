% -*- coding: utf-8 -*-
\documentclass{book}

% -*- coding: utf-8 -*-

\usepackage[b5paper,text={5in,8in},centering]{geometry}
\usepackage{amsmath}
\usepackage{amssymb}
\usepackage[heading = false, scheme = plain, fontset = fandol]{ctex}
% \usepackage[CJKchecksingle]{xeCJK}
% \setmainfont[Mapping=tex-text]{TeX Gyre Schola}
\setmainfont[Mapping=tex-text]{TeX Gyre Pagella}
%\setsansfont{URW Gothic L Book}
%\setmonofont{Nimbus Mono L}
% \setCJKmainfont[BoldFont=FandolHei,ItalicFont=FandolKai]{FandolSong}
% \setCJKsansfont{FandolHei}
% \setCJKmonofont{FandolFang}
\xeCJKsetup{PunctStyle = kaiming}

\linespread{1.25}
\setlength{\parindent}{2em}
\setlength{\parskip}{0.5ex}
\usepackage{indentfirst}

\usepackage{xcolor}
\definecolor{myblue}{rgb}{0,0.2,0.6}

\usepackage{titlesec}
\titleformat{\chapter}
    {\normalfont\Huge\sffamily\color{myblue}}
    {第\thechapter 章}
    {1em}
    {}
%\titlespacing{\chapter}{0pt}{50pt}{40pt}
\titleformat{\section}
    {\normalfont\Large\sffamily\color{myblue}}
    {\thesection}
    {1em}
    {}
%\titlespacing{\section}{0pt}{3.5ex plus 1ex minus .2ex}{2.3ex plus .2ex}
\titleformat{\subsection}
    {\normalfont\large\sffamily\color{myblue}}
    {\thesubsection}
    {1em}
    {}
%\titlespacing{\subsection}{0pt}{3.25ex plus 1ex minus .2ex}{1.5ex plus .2ex}
%
\newpagestyle{special}[\small\sffamily]{
  \headrule
  \sethead[\usepage][][\chaptertitle]
  {\chaptertitle}{}{\usepage}}
\newpagestyle{main}[\small\sffamily]{
  \headrule
  \sethead[\usepage][][第\thechapter 章\quad\chaptertitle]
  {\thesection\quad\sectiontitle}{}{\usepage}}

\usepackage{titletoc}
%\setcounter{tocdepth}{1}
%\titlecontents{标题层次}[左间距]{上间距和整体格式}{标题序号}{标题内容}{指引线和页码}[下间距]
\titlecontents{chapter}[1.5em]{\vspace{.5em}\bfseries\sffamily}{\color{myblue}\contentslabel{1.5em}}{}
    {\titlerule*[20pt]{$\cdot$}\contentspage}[]
\titlecontents{section}[4.5em]{\sffamily}{\color{myblue}\contentslabel{3em}}{}
    {\titlerule*[20pt]{$\cdot$}\contentspage}[]
%\titlecontents{subsection}[8.5em]{\sffamily}{\contentslabel{4em}}{}
%    {\titlerule*[20pt]{$\cdot$}\contentspage}

\usepackage{enumitem}
\setlist{topsep=2pt,itemsep=2pt,parsep=1pt,leftmargin=\parindent}

\usepackage{fancyvrb}
\DefineVerbatimEnvironment{verbatim}{Verbatim}
  {xleftmargin=2em,baselinestretch=1,formatcom=\color{teal}\upshape}

\usepackage{etoolbox}
\makeatletter
\preto{\FV@ListVSpace}{\topsep=2pt \partopsep=0pt }
\makeatother

\usepackage[colorlinks,plainpages,pagebackref]{hyperref}
\hypersetup{
   pdfstartview={FitH},
   citecolor=teal,
   linkcolor=myblue,
   urlcolor=black,
   bookmarksnumbered
}
\makeatletter
\def\UrlAlphabet{%
      \do\a\do\b\do\c\do\d\do\e\do\f\do\g\do\h\do\i\do\j%
      \do\k\do\l\do\m\do\n\do\o\do\p\do\q\do\r\do\s\do\t%
      \do\u\do\v\do\w\do\x\do\y\do\z\do\A\do\B\do\C\do\D%
      \do\E\do\F\do\G\do\H\do\I\do\J\do\K\do\L\do\M\do\N%
      \do\O\do\P\do\Q\do\R\do\S\do\T\do\U\do\V\do\W\do\X%
      \do\Y\do\Z}
\def\UrlDigits{\do\1\do\2\do\3\do\4\do\5\do\6\do\7\do\8\do\9\do\0}
\g@addto@macro{\UrlBreaks}{\UrlOrds}
\g@addto@macro{\UrlBreaks}{\UrlAlphabet}
\g@addto@macro{\UrlBreaks}{\UrlDigits}
\makeatother

\usepackage{comment,makeidx,multicol}

%\usepackage{german}
%% german
%\righthyphenmin=3
%\mdqoff
%\captionsenglish
\usepackage[english]{babel}
{\catcode`"=13 \gdef"#1{\ifx#1"\discretionary{}{}{}\fi\relax}}
\def\mdqon{\catcode`"=13\relax}
\def\mdqoff{\catcode`"=12\relax}
\makeindex
\hyphenation{ex-em-pli-fies}

\newdimen\tempdima \newdimen\tempdimb

% these are fine
\def\handbreak{\\ \message{^^JManual break!!!!^^J}}
\def\nl{\protect\\}\def\n#1{{\tt #1}}
\protected\def\cs#1{\texttt{\textbackslash#1}}
\pdfstringdefDisableCommands{\def\cs#1{\textbackslash#1}}
\let\csc\cs
\def\lb{{\tt\char`\{}}\def\rb{{\tt\char`\}}}
\def\gr#1{\texorpdfstring{$\langle$#1$\rangle$}{<#1>}} %\def\gr#1{$\langle$#1$\rangle$}
\def\marg#1{{\tt \{}#1{\tt \}}}
\def\oarg#1{{\tt [}#1{\tt }}
\def\key#1{{\tt#1}}
\def\alt{}\def\altt{}%this way in manstijl
\def\ldash{\unskip\ ——\nobreak\ \ignorespaces}
\def\rdash{\unskip\nobreak\ ——\ \ignorespaces}
% check these
\def\hex{{\tt"}}
\def\ascii{{\sc ascii}}
\def\ebcdic{{\sc ebcdic}}
\def\IniTeX{Ini\TeX}\def\LamsTeX{LAMS\TeX}\def\VirTeX{Vir\TeX}
\def\AmsTeX{Ams\TeX}
\def\TeXbook{the \TeX\ book}\def\web{{\sc web}}
% needs major thinking
\newenvironment{myquote}{\list{}{%
    \topsep=2pt \partopsep=0pt%
    \leftmargin=\parindent \rightmargin=\parindent
    }\item[]}{\endlist}
\newenvironment{disp}{\begin{myquote}}{\end{myquote}}
\newenvironment{Disp}{\begin{myquote}}{\end{myquote}}
\newenvironment{tdisp}{\begin{myquote}}{\end{myquote}}
\newenvironment{example}{\begin{myquote}\noindent\itshape 例子:}{\end{myquote}}
\newenvironment{inventory}{\begin{description}\raggedright}{\end{description}}
\newenvironment{glossinventory}{\begin{description}}{\end{description}}
\def\gram#1{\gr{#1}}%???
\def\meta{\gr}% alias
%
% index
%
\def\indexterm#1{\emph{#1}\index{#1}}
\def\indextermsub#1#2{\emph{#1 #2}\index{#1!#2}}
\def\indextermbus#1#2{\emph{#1 #2}\index{#2!#1}}
\def\cindextermsub#1#2{\emph{#1#2}\index{#1!#1#2}}
\def\cindextermbus#1#2{\emph{#1#2}\index{#2!#1#2}}
\def\term#1\par{\index{#1}}
\def\howto#1\par{}
\def\cstoidx#1\par{\index{#1@\cs{#1}@}}
\def\thecstoidx#1\par{\index{#1@\protect\csname #1\endcsname}}
\def\thecstoidxsub#1#2{\index{#1, #2@\protect\csname #1\endcsname, #2}\ignorespaces}
\def\csterm#1\par{\cstoidx #1\par\cs{#1}}
\def\csidx#1{\cstoidx #1\par\cs{#1}}

\def\tmc{\tracingmacros=2 \tracingcommands\tracingmacros}

%%%%%%%%%%%%%%%%%%%
\makeatletter
\def\snugbox{\hbox\bgroup\setbox\z@\vbox\bgroup
    \leftskip\z@
    \bgroup\aftergroup\make@snug
    \let\next=}
\def\make@snug{\par\sn@gify\egroup \box\z@\egroup}
\def\sn@gify
   {\skip\z@=\lastskip \unskip
    \advance\skip\z@\lastskip \unskip
    \unpenalty
    \setbox\z@\lastbox
    \ifvoid\z@ \nointerlineskip \else {\sn@gify} \fi
    \hbox{\unhbox\z@}\nointerlineskip
    \vskip\skip\z@
    }

\newdimen\fbh \fbh=60pt % dimension for easy scaling:
\newdimen\fbw \fbw=60pt % height and width of character box

\newdimen\dh \newdimen\dw % height and width of current character box
\newdimen\lh % height of previous character box
\newdimen\lw \lw=.4pt % line weight, instead of default .4pt

\def\hdotfill{\noindent
    \leaders\hbox{\vrule width 1pt height\lw
                  \kern4pt
                  \vrule width.5pt height\lw}\hfill\hbox{}
    \par}
\def\hlinefill{\noindent
    \leaders\hbox{\vrule width 5.5pt height\lw         }\hfill\hbox{}
    \par}
\def\stippel{$\qquad\qquad\qquad\qquad$}
\makeatother
%%%%%%%%%%%%%%%%%%%

%\def\SansSerif{\Typeface:macHelvetica }
%\def\SerifFont{\Typeface:macTimes }
%\def\SansSerif{\Typeface:bsGillSans }
%\def\SerifFont{\Typeface:bsBaskerville }
\let\SansSerif\relax \def\italic{\it}
\let\SerifFont\relax \def\MainFont{\rm}
\let\SansSerif\relax
\let\SerifFont\relax
\let\PopIndentLevel\relax \let\PushIndentLevel\relax
\let\ToVerso\relax \let\ToRecto\relax

%\def\stop@command@suffix{stop}
%\let\PopListLevel\PopIndentLevel
%\let\FlushRight\relax
%\let\flushright\FlushRight
%\let\SetListIndent\LevelIndent
%\def\awp{\ifhmode\vadjust{\penalty-10000 }\else
%    \penalty-10000 \fi}
\let\awp\relax
\let\PopIndentLevel\relax \let\PopListLevel\relax

\showboxdepth=-1

%\input figs
\def\endofchapter{\vfill\noindent}

\newcommand{\liamfnote}[1]{\protect\footnote{译注(Liam0205):#1}}
\newcommand{\cstate}[1]{状态 \textit{#1}}

\setcounter{chapter}{23}

\begin{document}

%\chapter{Display Math}\label{displaymath}
\chapter{Display Math}\label{displaymath}

%Displayed formulas are set on a line of their own, usually
%somewhere in a paragraph. This chapter explains
%how surrounding white space (both above/below and to the
%left/right) is calculated.
Displayed formulas are set on a line of their own, usually
somewhere in a paragraph. This chapter explains
how surrounding white space (both above/below and to the
left/right) is calculated.

%\label{cschap:abovedisplayskip}\label{cschap:belowdisplayskip}\label{cschap:abovedisplayshortskip}\label{cschap:belowdisplayshortskip}\label{cschap:predisplaypenalty}\label{cschap:postdisplaypenalty}\label{cschap:displayindent}\label{cschap:displaywidth}\label{cschap:predisplaysize}\label{cschap:everydisplay}\label{cschap:eqno}\label{cschap:leqno}
%\begin{inventory}
%\item [\cs{abovedisplayskip \cs{belowdisplayskip}}]
%\mdqon
%      Glue above/""below a display.
%\mdqoff
%      Plain \TeX\ default:~\n{12pt plus 3pt minus 9pt}
\label{cschap:abovedisplayskip}\label{cschap:belowdisplayskip}\label{cschap:abovedisplayshortskip}\label{cschap:belowdisplayshortskip}\label{cschap:predisplaypenalty}\label{cschap:postdisplaypenalty}\label{cschap:displayindent}\label{cschap:displaywidth}\label{cschap:predisplaysize}\label{cschap:everydisplay}\label{cschap:eqno}\label{cschap:leqno}
\begin{inventory}
\item [\cs{abovedisplayskip \cs{belowdisplayskip}}]
\mdqon
      Glue above/""below a display.
\mdqoff
      Plain \TeX\ default:~\n{12pt plus 3pt minus 9pt}

%\item [\cs{abovedisplayshortskip \cs{belowdisplayshortskip}}]
%\mdqon
%      Glue above/""below a display if the line preceding the display 
%\mdqoff
%      was short.
%      Plain \TeX\ defaults:~\n{0pt plus 3pt} and
%      \n{7pt plus 3pt minus 4pt} respectively.
\item [\cs{abovedisplayshortskip \cs{belowdisplayshortskip}}]
\mdqon
      Glue above/""below a display if the line preceding the display 
\mdqoff
      was short.
      Plain \TeX\ defaults:~\n{0pt plus 3pt} and
      \n{7pt plus 3pt minus 4pt} respectively.

%\item [\cs{predisplaypenalty \cs{postdisplaypenalty}}]
%\mdqon
%      Penalty placed in the vertical list above/""below a display.
%\mdqoff
%      Plain \TeX\ defaults:~\n{10$\,$000} and~\n{0}
%      respectively.
\item [\cs{predisplaypenalty \cs{postdisplaypenalty}}]
\mdqon
      Penalty placed in the vertical list above/""below a display.
\mdqoff
      Plain \TeX\ defaults:~\n{10$\,$000} and~\n{0}
      respectively.

%\item [\cs{displayindent}] 
%      Distance by which the box, in which the display 
%      is centred, is indented owing to hanging indentation.
\item [\cs{displayindent}] 
      Distance by which the box, in which the display 
      is centred, is indented owing to hanging indentation.

%\item [\cs{displaywidth}] 
%      Width of the box in which the display is centred.
\item [\cs{displaywidth}] 
      Width of the box in which the display is centred.

%\item [\cs{predisplaysize}] 
%      Effective width of the line preceding the display.
\item [\cs{predisplaysize}] 
      Effective width of the line preceding the display.

%\item [\cs{everydisplay}] 
%      Token list inserted at the start of a display.
\item [\cs{everydisplay}] 
      Token list inserted at the start of a display.

%\item [\cs{eqno}] 
%      Place a right equation number in a display formula.
\item [\cs{eqno}] 
      Place a right equation number in a display formula.

%\item [\cs{leqno}] 
%      Place a left equation number in a display formula.
\item [\cs{leqno}] 
      Place a left equation number in a display formula.

%\end{inventory}
\end{inventory}

%%\point Displays
%\section{Displays}
%\point Displays
\section{Displays}

%\TeX\ starts building a \indexterm{display math} formula
%when it encounters two
%math shift characters (characters of category~3\index{category!3},
%\verb>$>~in plain \TeX) in a row.
%Another such pair (possibly followed
%\alt by one optional space) indicates the end of the display.
\TeX\ starts building a \indexterm{display math} formula
when it encounters two
math shift characters (characters of category~3\index{category!3},
\verb>$>~in plain \TeX) in a row.
Another such pair (possibly followed
\alt by one optional space) indicates the end of the display.

%Math shift is a \gr{horizontal command}, but displays are only
%allowed in unrestricted horizontal mode
%(\verb>$$>~is an empty math formula in restricted horizontal mode).
%Displays themselves, however, are started in the
%surrounding (possibly internal) vertical mode in order to calculate
%quantities such as~\cs{prevgraf}; the result of the display is
%appended to the vertical list.
Math shift is a \gr{horizontal command}, but displays are only
allowed in unrestricted horizontal mode
(\verb>$$>~is an empty math formula in restricted horizontal mode).
Displays themselves, however, are started in the
surrounding (possibly internal) vertical mode in order to calculate
quantities such as~\cs{prevgraf}; the result of the display is
appended to the vertical list.

%The part of the paragraph above the display is broken into
%lines as an independent paragraph (but \cs{prevgraf} is
%carried over; see below), and the remainder of the
%paragraph is set, starting with an empty list and \cs{spacefactor}
%equal to~1000. 
%The \cs{everypar} tokens are not inserted for the part of the
%paragraph after the display, nor is \cs{parskip} glue inserted.
% 
%Right at the beginning of the display the \csidx{everydisplay}
%token list is inserted (but after the calculation of
%\cs{displayindent}, \cs{displaywidth}, and \cs{predisplaysize}).
%See page~\pageref{left:display} for an example of the use
%of \cs{everydisplay}.
The part of the paragraph above the display is broken into
lines as an independent paragraph (but \cs{prevgraf} is
carried over; see below), and the remainder of the
paragraph is set, starting with an empty list and \cs{spacefactor}
equal to~1000. 
The \cs{everypar} tokens are not inserted for the part of the
paragraph after the display, nor is \cs{parskip} glue inserted.
 
Right at the beginning of the display the \csidx{everydisplay}
token list is inserted (but after the calculation of
\cs{displayindent}, \cs{displaywidth}, and \cs{predisplaysize}).
See page~\pageref{left:display} for an example of the use
of \cs{everydisplay}.

%The page builder is exercised
%before the display 
%(but after the \cs{everydisplay} tokens have been inserted),
%and after the display finishes.
The page builder is exercised
before the display 
(but after the \cs{everydisplay} tokens have been inserted),
and after the display finishes.

%The `display style' of math typesetting was treated in 
%Chapter~\ref{mathfont}.
The `display style' of math typesetting was treated in 
Chapter~\ref{mathfont}.

%%\point Displays in paragraphs
%\section{Displays in paragraphs}
%\point Displays in paragraphs
\section{Displays in paragraphs}

%Positioning of a display in a paragraph may be influenced
%by hanging indentation or a \cs{parshape} specification.
%For this, \TeX\ uses the \cs{prevgraf} parameter
%(see Chapter~\ref{par:shape}), and
%acts as if the display is three lines deep.
Positioning of a display in a paragraph may be influenced
by hanging indentation or a \cs{parshape} specification.
For this, \TeX\ uses the \cs{prevgraf} parameter
(see Chapter~\ref{par:shape}), and
acts as if the display is three lines deep.

%If $n$ is the value of \cs{prevgraf} when the display starts
%\ldash so there are $n$ lines of text above the display \rdash 
%\cs{prevgraf} is set to to $n+3$ when the paragraph resumes.
%The display occupies, as it were, lines $n+1$, $n+2$, and~$n+3$.
%The shift and line width for the display are those
%that would hold for line~$n+2$.
If $n$ is the value of \cs{prevgraf} when the display starts
\ldash so there are $n$ lines of text above the display \rdash 
\cs{prevgraf} is set to to $n+3$ when the paragraph resumes.
The display occupies, as it were, lines $n+1$, $n+2$, and~$n+3$.
The shift and line width for the display are those
that would hold for line~$n+2$.

%The shift for the display is recorded in \cs{displayindent};
%\cstoidx displayindent\par\cstoidx displaywidth\par
%the line width is recorded in\handbreak \cs{displaywidth}. These parameters
%(and the \cs{predisplaysize} explained below)
%are set immediately after the \verb>$$> has been scanned.
%Usually they are equal to zero and \cs{hsize} respectively.
%The user can change the values of these parameters; 
%\TeX\ will use the
%values that hold after the math list 
%of the display has been processed.
The shift for the display is recorded in \cs{displayindent};
\cstoidx displayindent\par\cstoidx displaywidth\par
the line width is recorded in\handbreak \cs{displaywidth}. These parameters
(and the \cs{predisplaysize} explained below)
are set immediately after the \verb>$$> has been scanned.
Usually they are equal to zero and \cs{hsize} respectively.
The user can change the values of these parameters; 
\TeX\ will use the
values that hold after the math list 
of the display has been processed.

%A~display is vertical material, and therefore
%not influenced by settings of \cs{leftskip} and \cs{rightskip}.
A~display is vertical material, and therefore
not influenced by settings of \cs{leftskip} and \cs{rightskip}.

%%\point Vertical material around displays
%\section{Vertical material around displays}
%\point Vertical material around displays
\section{Vertical material around displays}

%A display is preceded in the vertical list by
%\begin{itemize}\item a penalty of size \cs{predisplaypenalty}
%   \cstoidx predisplaypenalty\par\cstoidx abovedisplayskip\par
%   \cstoidx abovedisplayshortskip\par
%(plain \TeX\ default~$10\,000$), and
%\item glue of size \cs{abovedisplayskip} 
%or \cs{abovedisplayshortskip}; this glue is omitted in
%cases where a~\cs{leqno} equation number is set on
%a line of its own (see below).\end{itemize}
%A display is followed by 
%\begin{itemize}\item a penalty of size \cs{postdisplaypenalty}
%   \cstoidx postdisplaypenalty\par\cstoidx belowdisplayskip\par
%   \cstoidx belowdisplayshortskip\par
%(default~0), and possibly
%\item glue of size \cs{belowdisplayskip} or 
%\cs{belowdisplayshortskip}; this glue is omitted in cases
%where an~\cs{eqno} equation number is set on a line of
%its own (see below).\end{itemize}
A display is preceded in the vertical list by
\begin{itemize}\item a penalty of size \cs{predisplaypenalty}
   \cstoidx predisplaypenalty\par\cstoidx abovedisplayskip\par
   \cstoidx abovedisplayshortskip\par
(plain \TeX\ default~$10\,000$), and
\item glue of size \cs{abovedisplayskip} 
or \cs{abovedisplayshortskip}; this glue is omitted in
cases where a~\cs{leqno} equation number is set on
a line of its own (see below).\end{itemize}
A display is followed by 
\begin{itemize}\item a penalty of size \cs{postdisplaypenalty}
   \cstoidx postdisplaypenalty\par\cstoidx belowdisplayskip\par
   \cstoidx belowdisplayshortskip\par
(default~0), and possibly
\item glue of size \cs{belowdisplayskip} or 
\cs{belowdisplayshortskip}; this glue is omitted in cases
where an~\cs{eqno} equation number is set on a line of
its own (see below).\end{itemize}

%The `short' variants of the glue are taken if
%there is no \cs{leqno} left equation number, and if
%the last line of the paragraph above the display is
%short enough for the display to be raised a bit without
%coming too close to that line.
%In order to decide this, the effective width of the
%preceding line is saved in \csidx{predisplaysize}.
%This value is calculated immediately after the opening \verb>$$>
%of the display has
%been scanned, together with the \cs{displaywidth}
%and \cs{displayindent} explained above.
The `short' variants of the glue are taken if
there is no \cs{leqno} left equation number, and if
the last line of the paragraph above the display is
short enough for the display to be raised a bit without
coming too close to that line.
In order to decide this, the effective width of the
preceding line is saved in \csidx{predisplaysize}.
This value is calculated immediately after the opening \verb>$$>
of the display has
been scanned, together with the \cs{displaywidth}
and \cs{displayindent} explained above.

%Remembering that the part of the paragraph above the display
%has already been broken into lines, the following method
%for finding the effective width of the last line ensues.
%\TeX\ takes the last box of the list, which is a horizontal
%box containing the last line, and locates the right edge
%of the last box in it. The \cs{predisplaysize} is then
%the place of that rightmost edge, plus any amount by which
%the last line was shifted, plus two ems in the current font.
Remembering that the part of the paragraph above the display
has already been broken into lines, the following method
for finding the effective width of the last line ensues.
\TeX\ takes the last box of the list, which is a horizontal
box containing the last line, and locates the right edge
of the last box in it. The \cs{predisplaysize} is then
the place of that rightmost edge, plus any amount by which
the last line was shifted, plus two ems in the current font.

%There are two exceptions to this. The \cs{predisplaysize}
%is taken to be $-$\cs{maxdimen} if there was no previous line,
%that is,
%the display started the paragraph, or it followed another display;
%\cs{predisplaysize} is taken to be \cs{maxdimen}
%if the glue in the last line was not set at its natural width,
%which may happen if the \cs{parfillskip} contained only finite
%stretch. The reason for the last clause is that glue
%\mdqon
%setting is slightly \emph{machine"-dependent}\index{machine dependence},
%and such dependences
%\mdqoff
%should be kept out of \TeX's global decision processes.
There are two exceptions to this. The \cs{predisplaysize}
is taken to be $-$\cs{maxdimen} if there was no previous line,
that is,
the display started the paragraph, or it followed another display;
\cs{predisplaysize} is taken to be \cs{maxdimen}
if the glue in the last line was not set at its natural width,
which may happen if the \cs{parfillskip} contained only finite
stretch. The reason for the last clause is that glue
\mdqon
setting is slightly \emph{machine"-dependent}\index{machine dependence},
and such dependences
\mdqoff
should be kept out of \TeX's global decision processes.

%%\point Glue setting of the display math list
%\section{Glue setting of the display math list}
%\point Glue setting of the display math list
\section{Glue setting of the display math list}

%The display has to fit in \cs{displaywidth}, 
%but in addition to the formula there
%may be an equation number. The minimum separation
%between the formula and the equation number should
%be one em in the symbol font, that is,
%\mdqon
%\cs{font\-dimen\-6}""\cs{textfont2}.
%\mdqoff
The display has to fit in \cs{displaywidth}, 
but in addition to the formula there
may be an equation number. The minimum separation
between the formula and the equation number should
be one em in the symbol font, that is,
\mdqon
\cs{font\-dimen\-6}""\cs{textfont2}.
\mdqoff

%If the formula plus any equation number
%and separation fit into \cs{displaywidth},
%the glue in the formula is set at its natural width. 
%If it does not fit,
%but the formula contains enough shrink, it is shrunk.
%Otherwise \TeX\ puts any equation number
%on a line of its own, and the glue in the formula is
%set to fit it in \cs{displaywidth}.
%With the equation
%number on a separate line the formula may now very well fit in the
%display width; however,
%if it was a very long formula the box in which it is
%set may still be overfull. \TeX\ nevers breaks a displayed
%formula.
If the formula plus any equation number
and separation fit into \cs{displaywidth},
the glue in the formula is set at its natural width. 
If it does not fit,
but the formula contains enough shrink, it is shrunk.
Otherwise \TeX\ puts any equation number
on a line of its own, and the glue in the formula is
set to fit it in \cs{displaywidth}.
With the equation
number on a separate line the formula may now very well fit in the
display width; however,
if it was a very long formula the box in which it is
set may still be overfull. \TeX\ nevers breaks a displayed
formula.

%%\point Centring the display formula: displacement
%\section{Centring the display formula: displacement}
%\point Centring the display formula: displacement
\section{Centring the display formula: displacement}

%Based on the width of the box containing the formula \ldash which
%may not really `contain' it; it may be overfull \rdash 
%\TeX\ tries to centre the formula in the \cs{displaywidth},
%that is, without taking the equation number into account.
%Initially, a displacement is calculated that is 
%half the difference between \cs{displaywidth} and the
%width of the formula box.
Based on the width of the box containing the formula \ldash which
may not really `contain' it; it may be overfull \rdash 
\TeX\ tries to centre the formula in the \cs{displaywidth},
that is, without taking the equation number into account.
Initially, a displacement is calculated that is 
half the difference between \cs{displaywidth} and the
width of the formula box.

%However, if there is an equation number that will not
%be put on a separate line and the displacement is less than
%twice the width of the equation number, a new displacement
%is calculated. This new displacement is zero if the formula
%started with glue; otherwise it is such that the
%formula box is centred in the space left by the equation
%number.
However, if there is an equation number that will not
be put on a separate line and the displacement is less than
twice the width of the equation number, a new displacement
is calculated. This new displacement is zero if the formula
started with glue; otherwise it is such that the
formula box is centred in the space left by the equation
number.

%If there was no equation number, or if the equation number
%will be put on a separate line, the formula box
%is now placed, shifted right by \cs{displayindent} plus
%the displacement calculated above.
If there was no equation number, or if the equation number
will be put on a separate line, the formula box
is now placed, shifted right by \cs{displayindent} plus
the displacement calculated above.

%\section{Equation numbers}
\section{Equation numbers}

%\index{equation numbering|(}
\index{equation numbering|(}

%The user can specify a equation number for a display
%by ending it with 
%\cstoidx eqno\par\cstoidx leqno\par
%\begin{Disp}\cs{eqno}\gr{math mode material}\verb>$$>\end{Disp}
%for an equation number placed on the right, or
%\begin{Disp}\cs{leqno}\gr{math mode material}\verb>$$>\end{Disp}
%for an equation number placed on the left.
The user can specify a equation number for a display
by ending it with 
\cstoidx eqno\par\cstoidx leqno\par
\begin{Disp}\cs{eqno}\gr{math mode material}\verb>$$>\end{Disp}
for an equation number placed on the right, or
\begin{Disp}\cs{leqno}\gr{math mode material}\verb>$$>\end{Disp}
for an equation number placed on the left.

%%\spoint Ordinary equation numbers
%\subsection{Ordinary equation numbers}
%\spoint Ordinary equation numbers
\subsection{Ordinary equation numbers}

%Above it was described how \TeX\ calculates a displacement
%from the display formula and the equation number, if this
%is to be put on the same line as the formula.
Above it was described how \TeX\ calculates a displacement
from the display formula and the equation number, if this
is to be put on the same line as the formula.

%If the equation number was a  \cs{leqno} number,
%\TeX\ places a box containing
%\begin{itemize}\item the equation number,
%\item a kern with the size of the displacement calculated, and
%\item the formula.\end{itemize}
%This box is shifted right by \cs{displayindent}.
If the equation number was a  \cs{leqno} number,
\TeX\ places a box containing
\begin{itemize}\item the equation number,
\item a kern with the size of the displacement calculated, and
\item the formula.\end{itemize}
This box is shifted right by \cs{displayindent}.

%If the equation number was an \cs{eqno} number,
%\TeX\ places a box containing
%\begin{itemize}\item the formula,
%\item a kern with the size of the displacement calculated, and
%\item the equation number.\end{itemize}
%This box is shifted right by \cs{displayindent} plus
%the displacement calculated.
If the equation number was an \cs{eqno} number,
\TeX\ places a box containing
\begin{itemize}\item the formula,
\item a kern with the size of the displacement calculated, and
\item the equation number.\end{itemize}
This box is shifted right by \cs{displayindent} plus
the displacement calculated.

%%\spoint The equation number on a separate line
%\subsection{The equation number on a separate line}
%\spoint The equation number on a separate line
\subsection{The equation number on a separate line}

%Since displayed formulas may become rather big, \TeX\ can decide
%(as was described above)
%that any equation number should be placed on a line of its own.
%A~left-placed equation number is then to be placed above the
%display, in a box that is shifted right by \cs{displayindent};
%a right-placed equation number will be placed below the display,
%in a box that is shifted to the right 
%by \cs{displayindent} plus \cs{displaywidth} minus the width of
%the equation number box.
Since displayed formulas may become rather big, \TeX\ can decide
(as was described above)
that any equation number should be placed on a line of its own.
A~left-placed equation number is then to be placed above the
display, in a box that is shifted right by \cs{displayindent};
a right-placed equation number will be placed below the display,
in a box that is shifted to the right 
by \cs{displayindent} plus \cs{displaywidth} minus the width of
the equation number box.

%In both cases a penalty of $10\,000$ is placed between the equation
%number box and the formula.
In both cases a penalty of $10\,000$ is placed between the equation
number box and the formula.

%\TeX\ does not put extra glue above a left-placed
%equation number or below
%a right-placed equation number; \TeX\ here relies on
%the baselineskip mechanism.
\TeX\ does not put extra glue above a left-placed
equation number or below
a right-placed equation number; \TeX\ here relies on
the baselineskip mechanism.

%\index{equation numbering|)}
\index{equation numbering|)}

%\section{Non-centred displays}
%\label{left:display}
\section{Non-centred displays}
\label{left:display}

%As a default, \TeX\ will center displays.
%In order to get \indextermbus{non-centred}{displays} some
%macro trickery is needed. 
As a default, \TeX\ will center displays.
In order to get \indextermbus{non-centred}{displays} some
macro trickery is needed. 

%One approach would
%be to write a macro \cs{DisplayEquation}
%that would basically look like
%\begin{verbatim}
%\def\DisplayEquation#1{%
%    \par \vskip\abovedisplayskip
%    \hbox{\kern\parindent$\displaystyle#1$}
%    \vskip\belowdisplayskip \noindent}
%\end{verbatim}
%but it would be nicer if one could just write
%\begin{verbatim}
%$$ ... \eqno ... $$
%\end{verbatim}
%\mdqon
%and having this come out as a left"-aligning display.
%\mdqoff
One approach would
be to write a macro \cs{DisplayEquation}
that would basically look like
\begin{verbatim}
\def\DisplayEquation#1{%
    \par \vskip\abovedisplayskip
    \hbox{\kern\parindent$\displaystyle#1$}
    \vskip\belowdisplayskip \noindent}
\end{verbatim}
but it would be nicer if one could just write
\begin{verbatim}
$$ ... \eqno ... $$
\end{verbatim}
\mdqon
and having this come out as a left"-aligning display.
\mdqoff

%Using the \cs{everydisplay} token list, the above
%idea can be realized. The basic idea is to write
%\begin{verbatim}
%\everydisplay{\IndentedDisplay}
%\def\IndentedDisplay#1$${ ...
%\end{verbatim}
%so that the macro \cs{IndentedDisplay}
%will receive the formula, including any equation number.
%The first step is now to extract an equation number
%if it is present. This makes creative use of delimited
%macro parameters.
%\begin{verbatim}
%\def\ExtractEqNo#1\eqno#2\eqno#3\relax
%   {\def\Equation{#1}\def\EqNo{#2}}
%\def\IndentedDisplay#1$${%
%    \ExtractEqNo#1\eqno\eqno\relax
%\end{verbatim}
%Next the equation should be set in the available
%space \cs{displaywidth}:
%\begin{verbatim}
%    \hbox to \displaywidth
%        {\kern\parindent
%         $\displaystyle\Equation$\hfil$\EqNo$}$$
%    }
%\end{verbatim}
%Note that the macro ends in the closing \verb>$$>
%to balance the opening dollars that caused
%insertion of the \cs{everydisplay} tokens.
%This also means that the box containing the
%displayed material will automatically be
%surrounded by \cs{abovedisplayskip} and
%\cs{belowdisplayskip} glue.
%There is no need to use \cs{displayindent} anywhere
%in this macro, because \TeX\ itself will shift the
%display appropriately.
Using the \cs{everydisplay} token list, the above
idea can be realized. The basic idea is to write
\begin{verbatim}
\everydisplay{\IndentedDisplay}
\def\IndentedDisplay#1$${ ...
\end{verbatim}
so that the macro \cs{IndentedDisplay}
will receive the formula, including any equation number.
The first step is now to extract an equation number
if it is present. This makes creative use of delimited
macro parameters.
\begin{verbatim}
\def\ExtractEqNo#1\eqno#2\eqno#3\relax
   {\def\Equation{#1}\def\EqNo{#2}}
\def\IndentedDisplay#1$${%
    \ExtractEqNo#1\eqno\eqno\relax
\end{verbatim}
Next the equation should be set in the available
space \cs{displaywidth}:
\begin{verbatim}
    \hbox to \displaywidth
        {\kern\parindent
         $\displaystyle\Equation$\hfil$\EqNo$}$$
    }
\end{verbatim}
Note that the macro ends in the closing \verb>$$>
to balance the opening dollars that caused
insertion of the \cs{everydisplay} tokens.
This also means that the box containing the
displayed material will automatically be
surrounded by \cs{abovedisplayskip} and
\cs{belowdisplayskip} glue.
There is no need to use \cs{displayindent} anywhere
in this macro, because \TeX\ itself will shift the
display appropriately.

%% 
%\endofchapter
% 
\endofchapter

\end{document}
