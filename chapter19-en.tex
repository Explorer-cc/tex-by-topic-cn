% -*- coding: utf-8 -*-
\documentclass{book}

% -*- coding: utf-8 -*-

\usepackage[b5paper,text={5in,8in},centering]{geometry}
\usepackage{amsmath}
\usepackage{amssymb}
\usepackage[heading = false, scheme = plain, fontset = fandol]{ctex}
% \usepackage[CJKchecksingle]{xeCJK}
% \setmainfont[Mapping=tex-text]{TeX Gyre Schola}
\setmainfont[Mapping=tex-text]{TeX Gyre Pagella}
%\setsansfont{URW Gothic L Book}
%\setmonofont{Nimbus Mono L}
% \setCJKmainfont[BoldFont=FandolHei,ItalicFont=FandolKai]{FandolSong}
% \setCJKsansfont{FandolHei}
% \setCJKmonofont{FandolFang}
\xeCJKsetup{PunctStyle = kaiming}

\linespread{1.25}
\setlength{\parindent}{2em}
\setlength{\parskip}{0.5ex}
\usepackage{indentfirst}

\usepackage{xcolor}
\definecolor{myblue}{rgb}{0,0.2,0.6}

\usepackage{titlesec}
\titleformat{\chapter}
    {\normalfont\Huge\sffamily\color{myblue}}
    {第\thechapter 章}
    {1em}
    {}
%\titlespacing{\chapter}{0pt}{50pt}{40pt}
\titleformat{\section}
    {\normalfont\Large\sffamily\color{myblue}}
    {\thesection}
    {1em}
    {}
%\titlespacing{\section}{0pt}{3.5ex plus 1ex minus .2ex}{2.3ex plus .2ex}
\titleformat{\subsection}
    {\normalfont\large\sffamily\color{myblue}}
    {\thesubsection}
    {1em}
    {}
%\titlespacing{\subsection}{0pt}{3.25ex plus 1ex minus .2ex}{1.5ex plus .2ex}
%
\newpagestyle{special}[\small\sffamily]{
  \headrule
  \sethead[\usepage][][\chaptertitle]
  {\chaptertitle}{}{\usepage}}
\newpagestyle{main}[\small\sffamily]{
  \headrule
  \sethead[\usepage][][第\thechapter 章\quad\chaptertitle]
  {\thesection\quad\sectiontitle}{}{\usepage}}

\usepackage{titletoc}
%\setcounter{tocdepth}{1}
%\titlecontents{标题层次}[左间距]{上间距和整体格式}{标题序号}{标题内容}{指引线和页码}[下间距]
\titlecontents{chapter}[1.5em]{\vspace{.5em}\bfseries\sffamily}{\color{myblue}\contentslabel{1.5em}}{}
    {\titlerule*[20pt]{$\cdot$}\contentspage}[]
\titlecontents{section}[4.5em]{\sffamily}{\color{myblue}\contentslabel{3em}}{}
    {\titlerule*[20pt]{$\cdot$}\contentspage}[]
%\titlecontents{subsection}[8.5em]{\sffamily}{\contentslabel{4em}}{}
%    {\titlerule*[20pt]{$\cdot$}\contentspage}

\usepackage{enumitem}
\setlist{topsep=2pt,itemsep=2pt,parsep=1pt,leftmargin=\parindent}

\usepackage{fancyvrb}
\DefineVerbatimEnvironment{verbatim}{Verbatim}
  {xleftmargin=2em,baselinestretch=1,formatcom=\color{teal}\upshape}

\usepackage{etoolbox}
\makeatletter
\preto{\FV@ListVSpace}{\topsep=2pt \partopsep=0pt }
\makeatother

\usepackage[colorlinks,plainpages,pagebackref]{hyperref}
\hypersetup{
   pdfstartview={FitH},
   citecolor=teal,
   linkcolor=myblue,
   urlcolor=black,
   bookmarksnumbered
}
\makeatletter
\def\UrlAlphabet{%
      \do\a\do\b\do\c\do\d\do\e\do\f\do\g\do\h\do\i\do\j%
      \do\k\do\l\do\m\do\n\do\o\do\p\do\q\do\r\do\s\do\t%
      \do\u\do\v\do\w\do\x\do\y\do\z\do\A\do\B\do\C\do\D%
      \do\E\do\F\do\G\do\H\do\I\do\J\do\K\do\L\do\M\do\N%
      \do\O\do\P\do\Q\do\R\do\S\do\T\do\U\do\V\do\W\do\X%
      \do\Y\do\Z}
\def\UrlDigits{\do\1\do\2\do\3\do\4\do\5\do\6\do\7\do\8\do\9\do\0}
\g@addto@macro{\UrlBreaks}{\UrlOrds}
\g@addto@macro{\UrlBreaks}{\UrlAlphabet}
\g@addto@macro{\UrlBreaks}{\UrlDigits}
\makeatother

\usepackage{comment,makeidx,multicol}

%\usepackage{german}
%% german
%\righthyphenmin=3
%\mdqoff
%\captionsenglish
\usepackage[english]{babel}
{\catcode`"=13 \gdef"#1{\ifx#1"\discretionary{}{}{}\fi\relax}}
\def\mdqon{\catcode`"=13\relax}
\def\mdqoff{\catcode`"=12\relax}
\makeindex
\hyphenation{ex-em-pli-fies}

\newdimen\tempdima \newdimen\tempdimb

% these are fine
\def\handbreak{\\ \message{^^JManual break!!!!^^J}}
\def\nl{\protect\\}\def\n#1{{\tt #1}}
\protected\def\cs#1{\texttt{\textbackslash#1}}
\pdfstringdefDisableCommands{\def\cs#1{\textbackslash#1}}
\let\csc\cs
\def\lb{{\tt\char`\{}}\def\rb{{\tt\char`\}}}
\def\gr#1{\texorpdfstring{$\langle$#1$\rangle$}{<#1>}} %\def\gr#1{$\langle$#1$\rangle$}
\def\marg#1{{\tt \{}#1{\tt \}}}
\def\oarg#1{{\tt [}#1{\tt }}
\def\key#1{{\tt#1}}
\def\alt{}\def\altt{}%this way in manstijl
\def\ldash{\unskip\ ——\nobreak\ \ignorespaces}
\def\rdash{\unskip\nobreak\ ——\ \ignorespaces}
% check these
\def\hex{{\tt"}}
\def\ascii{{\sc ascii}}
\def\ebcdic{{\sc ebcdic}}
\def\IniTeX{Ini\TeX}\def\LamsTeX{LAMS\TeX}\def\VirTeX{Vir\TeX}
\def\AmsTeX{Ams\TeX}
\def\TeXbook{the \TeX\ book}\def\web{{\sc web}}
% needs major thinking
\newenvironment{myquote}{\list{}{%
    \topsep=2pt \partopsep=0pt%
    \leftmargin=\parindent \rightmargin=\parindent
    }\item[]}{\endlist}
\newenvironment{disp}{\begin{myquote}}{\end{myquote}}
\newenvironment{Disp}{\begin{myquote}}{\end{myquote}}
\newenvironment{tdisp}{\begin{myquote}}{\end{myquote}}
\newenvironment{example}{\begin{myquote}\noindent\itshape 例子:}{\end{myquote}}
\newenvironment{inventory}{\begin{description}\raggedright}{\end{description}}
\newenvironment{glossinventory}{\begin{description}}{\end{description}}
\def\gram#1{\gr{#1}}%???
\def\meta{\gr}% alias
%
% index
%
\def\indexterm#1{\emph{#1}\index{#1}}
\def\indextermsub#1#2{\emph{#1 #2}\index{#1!#2}}
\def\indextermbus#1#2{\emph{#1 #2}\index{#2!#1}}
\def\cindextermsub#1#2{\emph{#1#2}\index{#1!#1#2}}
\def\cindextermbus#1#2{\emph{#1#2}\index{#2!#1#2}}
\def\term#1\par{\index{#1}}
\def\howto#1\par{}
\def\cstoidx#1\par{\index{#1@\cs{#1}@}}
\def\thecstoidx#1\par{\index{#1@\protect\csname #1\endcsname}}
\def\thecstoidxsub#1#2{\index{#1, #2@\protect\csname #1\endcsname, #2}\ignorespaces}
\def\csterm#1\par{\cstoidx #1\par\cs{#1}}
\def\csidx#1{\cstoidx #1\par\cs{#1}}

\def\tmc{\tracingmacros=2 \tracingcommands\tracingmacros}

%%%%%%%%%%%%%%%%%%%
\makeatletter
\def\snugbox{\hbox\bgroup\setbox\z@\vbox\bgroup
    \leftskip\z@
    \bgroup\aftergroup\make@snug
    \let\next=}
\def\make@snug{\par\sn@gify\egroup \box\z@\egroup}
\def\sn@gify
   {\skip\z@=\lastskip \unskip
    \advance\skip\z@\lastskip \unskip
    \unpenalty
    \setbox\z@\lastbox
    \ifvoid\z@ \nointerlineskip \else {\sn@gify} \fi
    \hbox{\unhbox\z@}\nointerlineskip
    \vskip\skip\z@
    }

\newdimen\fbh \fbh=60pt % dimension for easy scaling:
\newdimen\fbw \fbw=60pt % height and width of character box

\newdimen\dh \newdimen\dw % height and width of current character box
\newdimen\lh % height of previous character box
\newdimen\lw \lw=.4pt % line weight, instead of default .4pt

\def\hdotfill{\noindent
    \leaders\hbox{\vrule width 1pt height\lw
                  \kern4pt
                  \vrule width.5pt height\lw}\hfill\hbox{}
    \par}
\def\hlinefill{\noindent
    \leaders\hbox{\vrule width 5.5pt height\lw         }\hfill\hbox{}
    \par}
\def\stippel{$\qquad\qquad\qquad\qquad$}
\makeatother
%%%%%%%%%%%%%%%%%%%

%\def\SansSerif{\Typeface:macHelvetica }
%\def\SerifFont{\Typeface:macTimes }
%\def\SansSerif{\Typeface:bsGillSans }
%\def\SerifFont{\Typeface:bsBaskerville }
\let\SansSerif\relax \def\italic{\it}
\let\SerifFont\relax \def\MainFont{\rm}
\let\SansSerif\relax
\let\SerifFont\relax
\let\PopIndentLevel\relax \let\PushIndentLevel\relax
\let\ToVerso\relax \let\ToRecto\relax

%\def\stop@command@suffix{stop}
%\let\PopListLevel\PopIndentLevel
%\let\FlushRight\relax
%\let\flushright\FlushRight
%\let\SetListIndent\LevelIndent
%\def\awp{\ifhmode\vadjust{\penalty-10000 }\else
%    \penalty-10000 \fi}
\let\awp\relax
\let\PopIndentLevel\relax \let\PopListLevel\relax

\showboxdepth=-1

%\input figs
\def\endofchapter{\vfill\noindent}

\newcommand{\liamfnote}[1]{\protect\footnote{译注(Liam0205):#1}}
\newcommand{\cstate}[1]{状态 \textit{#1}}

\setcounter{chapter}{18}

\begin{document}

%\chapter{Line Breaking}\label{line:break}
%\index{paragraph!breaking into lines|(}
%\index{line breaking|(}
\chapter{Line Breaking}\label{line:break}
\index{paragraph!breaking into lines|(}
\index{line breaking|(}

%This chapter treats line breaking and the concept of `badness' that \TeX\
%uses to decide how to break a paragraph into lines,
%or where to break a page.
%The various penalties contributing to the cost of line breaking
%are treated here, as is hyphenation.
%Page breaking is treated in Chapter~\ref{page:break}.
This chapter treats line breaking and the concept of `badness' that \TeX\
uses to decide how to break a paragraph into lines,
or where to break a page.
The various penalties contributing to the cost of line breaking
are treated here, as is hyphenation.
Page breaking is treated in Chapter~\ref{page:break}.

%\label{cschap:penalty}\label{cschap:linepenalty}\label{cschap:hyphenpenalty}\label{cschap:exhyphenpenalty}\label{cschap:adjdemerits}\label{cschap:doublehyphendemerits}\label{cschap:finalhyphendemerits}\label{cschap:allowbreak}\label{cschap:pretolerance}\label{cschap:tolerance}\label{cschap:emergencystretch}\label{cschap:looseness}\label{cschap:prevgraf}\label{cschap:discretionary}\label{cschap:-}\label{cschap:hyphenchar2}\label{cschap:defaulthyphenchar2}\label{cschap:uchyph}\label{cschap:lefthyphenmin}\label{cschap:righthyphenmin}\label{cschap:patterns}\label{cschap:hyphenation}\label{cschap:language}\label{cschap:setlanguage}
%\begin{inventory}
%\item [\cs{penalty}]
%      Specify desirability of not breaking at this point.
\label{cschap:penalty}\label{cschap:linepenalty}\label{cschap:hyphenpenalty}\label{cschap:exhyphenpenalty}\label{cschap:adjdemerits}\label{cschap:doublehyphendemerits}\label{cschap:finalhyphendemerits}\label{cschap:allowbreak}\label{cschap:pretolerance}\label{cschap:tolerance}\label{cschap:emergencystretch}\label{cschap:looseness}\label{cschap:prevgraf}\label{cschap:discretionary}\label{cschap:-}\label{cschap:hyphenchar2}\label{cschap:defaulthyphenchar2}\label{cschap:uchyph}\label{cschap:lefthyphenmin}\label{cschap:righthyphenmin}\label{cschap:patterns}\label{cschap:hyphenation}\label{cschap:language}\label{cschap:setlanguage}
\begin{inventory}
\item [\cs{penalty}]
      Specify desirability of not breaking at this point.

%\item [\cs{linepenalty}]
%      Penalty value associated with each line break.
%      Plain \TeX\ default:~\n{10}.
\item [\cs{linepenalty}]
      Penalty value associated with each line break.
      Plain \TeX\ default:~\n{10}.

%\item [\cs{hyphenpenalty}]
%      Penalty associated with break at a discretionary item
%      in the general case.
%      Plain \TeX\ default:~\n{50}.
\item [\cs{hyphenpenalty}]
      Penalty associated with break at a discretionary item
      in the general case.
      Plain \TeX\ default:~\n{50}.

%\item [\cs{exhyphenpenalty}]
%      Penalty for breaking a horizontal line at a discretionary
%      item in the special case where the prebreak text is empty.
%      Plain \TeX\ default:~\n{50}.
\item [\cs{exhyphenpenalty}]
      Penalty for breaking a horizontal line at a discretionary
      item in the special case where the prebreak text is empty.
      Plain \TeX\ default:~\n{50}.

%\item [\cs{adjdemerits}]
%      Penalty for adjacent visually incompatible lines.
%      Plain \TeX\ default:~\n{10$\,$000}.
\item [\cs{adjdemerits}]
      Penalty for adjacent visually incompatible lines.
      Plain \TeX\ default:~\n{10$\,$000}.

%\item [\cs{doublehyphendemerits}]
%      Penalty for consecutive lines ending with a hyphen.
%      Plain \TeX\ default: \n{10$\,$000}.
\item [\cs{doublehyphendemerits}]
      Penalty for consecutive lines ending with a hyphen.
      Plain \TeX\ default: \n{10$\,$000}.

%\item [\cs{finalhyphendemerits}]
%      Penalty added when the penultimate line of a
%      paragraph ends with a hyphen.
%      Plain \TeX\ default:~\n{5000}.
\item [\cs{finalhyphendemerits}]
      Penalty added when the penultimate line of a
      paragraph ends with a hyphen.
      Plain \TeX\ default:~\n{5000}.

%\item [\cs{allowbreak}]
%      Macro for creating a breakpoint by inserting a
%      \cs{penalty0}.
\item [\cs{allowbreak}]
      Macro for creating a breakpoint by inserting a
      \cs{penalty0}.

%\item [\cs{pretolerance}]
%      Tolerance value for a paragraph without hyphenation.
%      Plain \TeX\ default:~\n{100}.
\item [\cs{pretolerance}]
      Tolerance value for a paragraph without hyphenation.
      Plain \TeX\ default:~\n{100}.

%\item [\cs{tolerance}]
%      Tolerance value for lines in a paragraph with hyphenation.
%      Plain \TeX\ default:~\n{200}.
\item [\cs{tolerance}]
      Tolerance value for lines in a paragraph with hyphenation.
      Plain \TeX\ default:~\n{200}.

%\item [\cs{emergencystretch}]
%      (\TeX3 only)
%      Assumed extra stretchability in lines of a paragraph.
\item [\cs{emergencystretch}]
      (\TeX3 only)
      Assumed extra stretchability in lines of a paragraph.

%\item [\cs{looseness}]
%      Number of lines by which this paragraph has to be made longer
%      than it would be ideally.
\item [\cs{looseness}]
      Number of lines by which this paragraph has to be made longer
      than it would be ideally.

%\item [\cs{prevgraf}]
%      The number of lines in the paragraph last
%      added to the vertical list.
\item [\cs{prevgraf}]
      The number of lines in the paragraph last
      added to the vertical list.

%\item [\cs{discretionary}]
%      Specify the way a character sequence is split up at a line break.
\item [\cs{discretionary}]
      Specify the way a character sequence is split up at a line break.

%\item [\cs{-}]
%      Discretionary hyphen; this is
%      equivalent to \verb|\discretionary{-}{}{}|.
\item [\cs{-}]
      Discretionary hyphen; this is
      equivalent to \verb|\discretionary{-}{}{}|.

%\item [\cs{hyphenchar}]
%      Number of the hyphen character of a font.
\item [\cs{hyphenchar}]
      Number of the hyphen character of a font.

%\item [\cs{defaulthyphenchar}]
%      Value of \cs{hyphenchar} when a font is loaded.
%      Plain \TeX\ default:~\n{`\cs{-}}.
\item [\cs{defaulthyphenchar}]
      Value of \cs{hyphenchar} when a font is loaded.
      Plain \TeX\ default:~\n{`\cs{-}}.

%\item [\cs{uchyph}]
%      Positive to allow hyphenation of words starting with a capital
%      letter.
%      Plain \TeX\ default:~\n{1}.
\item [\cs{uchyph}]
      Positive to allow hyphenation of words starting with a capital
      letter.
      Plain \TeX\ default:~\n{1}.

%\item [\cs{lefthyphenmin}]
%      (\TeX3 only)
%      Minimal number of characters before a hyphenation.
%      Plain \TeX\ default:~\n{2}.
\item [\cs{lefthyphenmin}]
      (\TeX3 only)
      Minimal number of characters before a hyphenation.
      Plain \TeX\ default:~\n{2}.

%\item [\cs{righthyphenmin}]
%      (\TeX3 only)
%      Minimum number of characters after a hyphenation.
%      Plain \TeX\ default:~\n{3}.
\item [\cs{righthyphenmin}]
      (\TeX3 only)
      Minimum number of characters after a hyphenation.
      Plain \TeX\ default:~\n{3}.

%\item [\cs{patterns}]
%      Define a list of hyphenation patterns for the current
%      value of \cs{language};  allowed only in \IniTeX.
\item [\cs{patterns}]
      Define a list of hyphenation patterns for the current
      value of \cs{language};  allowed only in \IniTeX.

%\item [\cs{hyphenation}]
%      Define hyphenation exceptions for the current value of \cs{language}.
\item [\cs{hyphenation}]
      Define hyphenation exceptions for the current value of \cs{language}.

%\item [\cs{language}]
%      Choose a set of hyphenation patterns and exceptions.
\item [\cs{language}]
      Choose a set of hyphenation patterns and exceptions.

%\item [\cs{setlanguage}]
%      Reset the current language.
\item [\cs{setlanguage}]
      Reset the current language.

%\end{inventory}
\end{inventory}


%%\point Paragraph break cost calculation
%\section{Paragraph break cost calculation}
%\point Paragraph break cost calculation
\section{Paragraph break cost calculation}

%A paragraph is broken such that the amount $d$ of {\em demerits\/}
%associated with breaking it is minimized.
%The total amount of demerits for a paragraph is the sum
%of those for the individual lines, plus possibly some extra
%penalties. Considering a paragraph as a whole instead of
%breaking it on a line-by-line basis can lead to better
%line breaking: \TeX\ can choose to take a slightly less beautiful
%line in the beginning of the paragraph in order to avoid
%bigger trouble later on.
A paragraph is broken such that the amount $d$ of {\em demerits\/}
associated with breaking it is minimized.
The total amount of demerits for a paragraph is the sum
of those for the individual lines, plus possibly some extra
penalties. Considering a paragraph as a whole instead of
breaking it on a line-by-line basis can lead to better
line breaking: \TeX\ can choose to take a slightly less beautiful
line in the beginning of the paragraph in order to avoid
bigger trouble later on.

%For each line demerits are calculated from the {\em badness\/}~$b$
%of stretching or shrinking the line to the break, and
%the {\em penalty\/}~$p$ associated with the break.
%The badness is not allowed to exceed a certain prescribed
%tolerance.
For each line demerits are calculated from the {\em badness\/}~$b$
of stretching or shrinking the line to the break, and
the {\em penalty\/}~$p$ associated with the break.
The badness is not allowed to exceed a certain prescribed
tolerance.

%In addition to the demerits for breaking individual lines,
%\TeX\ assigns demerits for the way lines combine; see below.
In addition to the demerits for breaking individual lines,
\TeX\ assigns demerits for the way lines combine; see below.

%The
%\mdqon
%implementation of \TeX's paragraph"-breaking algorithm
%\mdqoff
%is explained in~\cite{K:break}.
The
\mdqon
implementation of \TeX's paragraph"-breaking algorithm
\mdqoff
is explained in~\cite{K:break}.

%%\spoint Badness
%\subsection{Badness}
%\spoint Badness
\subsection{Badness}

%From the ratio between the stretch or shrink present in a
%line, and the actual stretch or shrink taken,
%the \emph{badness}\index{badness! and line breaking}\index{line breaking!badness}
%of breaking a line at a certain point is calculated.
%This badness is an important
%factor in the process of line breaking.
%See page~\pageref{bad:form} for the formula for badness.
From the ratio between the stretch or shrink present in a
line, and the actual stretch or shrink taken,
the \emph{badness}\index{badness! and line breaking}\index{line breaking!badness}
of breaking a line at a certain point is calculated.
This badness is an important
factor in the process of line breaking.
See page~\pageref{bad:form} for the formula for badness.

%In this chapter
%badness will only be discussed in the context of line breaking.
%Badness is also computed when a vertical list is stretched
%or shrunk (see Chapter~\ref{page:break}).
In this chapter
badness will only be discussed in the context of line breaking.
Badness is also computed when a vertical list is stretched
or shrunk (see Chapter~\ref{page:break}).

%The following terminology is used to describe badness:
%\begin{description} \item [tight (3)]
%is any line that has shrunk with a badness~$b\geq13$,
%that is, by using at least one-half of its amount of shrink
%(see page~\pageref{bad:form} for the computation).
%\item [decent (2)]
%is any line with a badness~$b\leq12$.
%\item [loose (1)]
%is any line that has stretched with a badness~$b\geq13$,
%that is, by using at least one-half of its amount of stretch.
%\item [very loose (0)]
%is any line that has stretched with a badness~$b\geq100$,
%that is, by using its full amount of stretch or more. Recall
%that glue can stretch, but not shrink more than its
%allowed amount.
%\end{description}
%The numbering is used in trace output (Chapter~\ref{trace}), and
%it is also used in the following definition:
%if the classifications of two adjacent lines differ by more than~1,
%the lines are said to be {\em visually incompatible\/}.
%See below for the \cs{adjdemerits} parameter associated with this.
The following terminology is used to describe badness:
\begin{description} \item [tight (3)]
is any line that has shrunk with a badness~$b\geq13$,
that is, by using at least one-half of its amount of shrink
(see page~\pageref{bad:form} for the computation).
\item [decent (2)]
is any line with a badness~$b\leq12$.
\item [loose (1)]
is any line that has stretched with a badness~$b\geq13$,
that is, by using at least one-half of its amount of stretch.
\item [very loose (0)]
is any line that has stretched with a badness~$b\geq100$,
that is, by using its full amount of stretch or more. Recall
that glue can stretch, but not shrink more than its
allowed amount.
\end{description}
The numbering is used in trace output (Chapter~\ref{trace}), and
it is also used in the following definition:
if the classifications of two adjacent lines differ by more than~1,
the lines are said to be {\em visually incompatible\/}.
See below for the \cs{adjdemerits} parameter associated with this.

%Overfull horizontal and vertical
%boxes are passed unnoticed if their excess width
%or height is less than \cs{hfuzz} or \cs{vfuzz} respectively;
%they are not reported if the badness is less than
%\cs{hbadness} or \cs{vbadness} (see Chapter~\ref{boxes}).
Overfull horizontal and vertical
boxes are passed unnoticed if their excess width
or height is less than \cs{hfuzz} or \cs{vfuzz} respectively;
they are not reported if the badness is less than
\cs{hbadness} or \cs{vbadness} (see Chapter~\ref{boxes}).

%%\spoint Penalties and other break locations
%\subsection{Penalties and other break locations}
%
%Line breaks can occur at the following \emph{breakpoints}
%\index{lists !horizontal! breakpoints in}
%in horizontal lists:
%\begin{enumerate} \item At a penalty. The penalty value is the
%`aesthetic cost' of breaking the line at that place.
%Negative penalties are considered as bonuses.
%A~penalty of $10\,000$ or more inhibits, and a penalty
%of $-10\,000$ or less forces, a~break.
%\spoint Penalties and other break locations
\subsection{Penalties and other break locations}

Line breaks can occur at the following \emph{breakpoints}
\index{lists !horizontal! breakpoints in}
in horizontal lists:
\begin{enumerate} \item At a penalty. The penalty value is the
`aesthetic cost' of breaking the line at that place.
Negative penalties are considered as bonuses.
A~penalty of $10\,000$ or more inhibits, and a penalty
of $-10\,000$ or less forces, a~break.

%Putting more than one penalty
%in a row is equivalent to putting just the one with the
%minimal value, because that one is the best candidate for line breaking.
Putting more than one penalty
in a row is equivalent to putting just the one with the
minimal value, because that one is the best candidate for line breaking.

%Penalties in horizontal mode are inserted by the user (or a
%user macro). The only exception is the \cs{nobreak}
%inserted before the \cs{parfillskip} glue.
Penalties in horizontal mode are inserted by the user (or a
user macro). The only exception is the \cs{nobreak}
inserted before the \cs{parfillskip} glue.

%\item At a glue, if it is not part of a math formula, and
%if it is preceded by a non-discardable item (see Chapter~\ref{hvmode}).
%There is no penalty associated with breaking at glue.
\item At a glue, if it is not part of a math formula, and
if it is preceded by a non-discardable item (see Chapter~\ref{hvmode}).
There is no penalty associated with breaking at glue.

%The condition about the non-discardable precursor is necessary,
%because otherwise breaking in between  two pieces of glue would
%be possible, which would cause ragged edges to the paragraph.
The condition about the non-discardable precursor is necessary,
because otherwise breaking in between  two pieces of glue would
be possible, which would cause ragged edges to the paragraph.

%\item At a kern, if it is not part of a math formula
%and if it is followed by  glue.
%There is no penalty associated with breaking at a~kern.
\item At a kern, if it is not part of a math formula
and if it is followed by  glue.
There is no penalty associated with breaking at a~kern.

%\item At a math-off, if that is followed by glue.
%Since math-off
%(and math-on) act as kerns (see Chapter~\ref{math}),
%this is very much like the previous case.
%There is no penalty associated with breaking at a~math-off.
\item At a math-off, if that is followed by glue.
Since math-off
(and math-on) act as kerns (see Chapter~\ref{math}),
this is very much like the previous case.
There is no penalty associated with breaking at a~math-off.

%\item At a discretionary break. The penalty
%is the \cs{hyphenpenalty} or the \cs{exhyphenpenalty}.
%This is treated below.
%\end{enumerate}
\item At a discretionary break. The penalty
is the \cs{hyphenpenalty} or the \cs{exhyphenpenalty}.
This is treated below.
\end{enumerate}

%Any discardable material following the break \ldash glue, kerns,
%\mdqon
%math-on/""off and penalties \rdash  is discarded. If one considers
%\mdqoff
%a line break at glue (kern, math-on{/}off) to occur at the
%front end of the glue item, this implies that that piece
%of glue disappears in the break.
Any discardable material following the break \ldash glue, kerns,
\mdqon
math-on/""off and penalties \rdash  is discarded. If one considers
\mdqoff
a line break at glue (kern, math-on{/}off) to occur at the
front end of the glue item, this implies that that piece
of glue disappears in the break.

%%\spoint Demerits
%\subsection{Demerits}
%\spoint Demerits
\subsection{Demerits}

%From the badness of a line and the penalty, if any, the demerits
%of the line are calculated. Let $l$ be the value of
%\csidx{linepenalty}, $b$~the badness of the line,
%$p$~the penalty at the break; then the \indexterm{demerits}~$d$
%are given by
%\begin{disp}$\displaystyle d=\cases{(l+b)^2+p^2&if $0\leq p<10\,000$\cr
%           (l+b)^2-p^2&if $-10\,000<p<0$\cr
%           (l+b)^2    &if $p\leq-10\,000$\cr}$\end{disp}
From the badness of a line and the penalty, if any, the demerits
of the line are calculated. Let $l$ be the value of
\csidx{linepenalty}, $b$~the badness of the line,
$p$~the penalty at the break; then the \indexterm{demerits}~$d$
are given by
\begin{disp}$\displaystyle d=
\begin{cases} (l+b)^2+p^2& \text{if } 0\leq p<10\,000 \\
           (l+b)^2-p^2&\text{if } -10\,000<p<0 \\
           (l+b)^2    &\text{if } p\leq-10\,000 \end{cases}$\end{disp}

%Both this formula and the one for the badness are described
%\alt
%in \cite{K:break} as `quite arbitrary',
%but they have been shown  to lead to
%good results in practice.
Both this formula and the one for the badness are described
\alt
in \cite{K:break} as `quite arbitrary',
but they have been shown  to lead to
good results in practice.

%The demerits for a paragraph are the sum of the demerits for
%the lines, plus
%\begin{itemize}
%\item the \csidx{adjdemerits} for any two
%      adjacent lines that are not visually compatible (see above),
%\item \csidx{doublehyphendemerits} for any two
%      consecutive lines ending with a hyphen, and the
%\item \csidx{finalhyphendemerits}
%      if the penultimate line of a paragraph
%   ends with a hyphen.\end{itemize}
The demerits for a paragraph are the sum of the demerits for
the lines, plus
\begin{itemize}
\item the \csidx{adjdemerits} for any two
      adjacent lines that are not visually compatible (see above),
\item \csidx{doublehyphendemerits} for any two
      consecutive lines ending with a hyphen, and the
\item \csidx{finalhyphendemerits}
      if the penultimate line of a paragraph
   ends with a hyphen.\end{itemize}

%At the start of a paragraph \TeX\ acts as if
%there was a preceding line which was `decent'.
%Therefore \cs{adjdemerits} will be added if the first
%line is `very loose'. Also, the last line
%of a paragraph is ordinarily also `decent'
%\ldash all spaces are set at natural width
%owing to the infinite stretch in the \cs{parfillskip} \rdash
%so \cs{adjdemerits} are added if
%the preceding line is `very loose'.
At the start of a paragraph \TeX\ acts as if
there was a preceding line which was `decent'.
Therefore \cs{adjdemerits} will be added if the first
line is `very loose'. Also, the last line
of a paragraph is ordinarily also `decent'
\ldash all spaces are set at natural width
owing to the infinite stretch in the \cs{parfillskip} \rdash
so \cs{adjdemerits} are added if
the preceding line is `very loose'.

%Note that the penalties at which a line break
%is chosen weigh about as heavily as the badness of
%the line, so they can be relatively small.
%However, the three extra demerit parameters
%have to be of the order of the square of
%penalties and badnesses to weigh equally heavily.
Note that the penalties at which a line break
is chosen weigh about as heavily as the badness of
the line, so they can be relatively small.
However, the three extra demerit parameters
have to be of the order of the square of
penalties and badnesses to weigh equally heavily.

%%\spoint The number of lines of a paragraph
%\subsection{The number of lines of a paragraph}
%\spoint The number of lines of a paragraph
\subsection{The number of lines of a paragraph}

%After a paragraph has been completed (or partially
%completed prior to a display), the variable \csidx{prevgraf}
%records the number of lines in the paragraph.
%By assigning to this variable \ldash and
%because this is a \gr{special integer}
%such an assignment is automatically global \rdash
%\TeX's decision processes can be influenced.
%This may be useful in combination with hanging indentation
%or \cs{parshape} specifications (see Chapter~\ref{par:shape}).
After a paragraph has been completed (or partially
completed prior to a display), the variable \csidx{prevgraf}
records the number of lines in the paragraph.
By assigning to this variable \ldash and
because this is a \gr{special integer}
such an assignment is automatically global \rdash
\TeX's decision processes can be influenced.
This may be useful in combination with hanging indentation
or \cs{parshape} specifications (see Chapter~\ref{par:shape}).

%\mdqon
%Some direct influence of the line"-breaking process
%\mdqoff
%on the resulting number of lines exists. One factor
%is the \cs{linepenalty} which is included in the demerits
%of each line. By increasing the line penalty \TeX\ can be
%made to minimize the number of lines in a paragraph.
\mdqon
Some direct influence of the line"-breaking process
\mdqoff
on the resulting number of lines exists. One factor
is the \cs{linepenalty} which is included in the demerits
of each line. By increasing the line penalty \TeX\ can be
made to minimize the number of lines in a paragraph.

%Deviations from the optimal number of lines, that is, the
%number of lines stemming from the optimal way of breaking a
%paragraph into lines, can be forced by the user by means
%of the \csidx{looseness} parameter. This parameter, which is
%reset every time the shape parameters
%are cleared (see Chapter~\ref{par:shape}),
%indicates by how many lines the current
%paragraph should be made longer than is optimal. A~negative
%value of \cs{looseness} will attempt to make the paragraph shorter
%by a number of lines that is the absolute value of the parameter.
Deviations from the optimal number of lines, that is, the
number of lines stemming from the optimal way of breaking a
paragraph into lines, can be forced by the user by means
of the \csidx{looseness} parameter. This parameter, which is
reset every time the shape parameters
are cleared (see Chapter~\ref{par:shape}),
indicates by how many lines the current
paragraph should be made longer than is optimal. A~negative
value of \cs{looseness} will attempt to make the paragraph shorter
by a number of lines that is the absolute value of the parameter.

%\TeX\ will still observe the values
%of \cs{pretolerance} and \cs{tolerance} (see below)
%when lengthening or shortening a paragraph under influence
%of \cs{looseness}.
%Therefore,
%\TeX\ will only lengthen or shorten a paragraph for as far
%as is possible without exceeding these parameters.
\TeX\ will still observe the values
of \cs{pretolerance} and \cs{tolerance} (see below)
when lengthening or shortening a paragraph under influence
of \cs{looseness}.
Therefore,
\TeX\ will only lengthen or shorten a paragraph for as far
as is possible without exceeding these parameters.


%%\spoint[between:lines] Between the lines
%\subsection{Between the lines}
%\label{between:lines}
%\spoint[between:lines] Between the lines
\subsection{Between the lines}
\label{between:lines}

%\TeX's
%paragraph mechanism packages lines into horizontal boxes
%that are appended to the surrounding vertical list.
%The resulting sequence of vertical items is then a
%repeating sequence of
%\begin{itemize}\item a box containing a line of text,
%\item possibly migrated vertical material (see page~\pageref{migrate}),
%\item a penalty item reflecting the cost of a page break
%      at that point, which is normally the \cs{interlinepenalty}
%      (see Chapter~\ref{page:break}), and
%\item interline glue, which is calculated automatically
%      on basis of the \cs{prevdepth} (see Chapter~\ref{baseline}).
%\end{itemize}
\TeX's
paragraph mechanism packages lines into horizontal boxes
that are appended to the surrounding vertical list.
The resulting sequence of vertical items is then a
repeating sequence of
\begin{itemize}\item a box containing a line of text,
\item possibly migrated vertical material (see page~\pageref{migrate}),
\item a penalty item reflecting the cost of a page break
      at that point, which is normally the \cs{interlinepenalty}
      (see Chapter~\ref{page:break}), and
\item interline glue, which is calculated automatically
      on basis of the \cs{prevdepth} (see Chapter~\ref{baseline}).
\end{itemize}

%%\point The process of breaking
%\section{The process of breaking}
%\point The process of breaking
\section{The process of breaking}

%\TeX\ tries to break paragraphs in such a way that
%the badness of each line does not exceed a certain tolerance.
%If there exists more than one solution to this, the one with
%the fewest demerits is taken.
\TeX\ tries to break paragraphs in such a way that
the badness of each line does not exceed a certain tolerance.
If there exists more than one solution to this, the one with
the fewest demerits is taken.

%By setting \csidx{tracingparagraphs} to a positive value,
%\TeX\ can be made to report the calculations of the
%paragraph mechanism in the log file. Some implementations of \TeX\
%may have this option disabled to make \TeX\ run faster.
By setting \csidx{tracingparagraphs} to a positive value,
\TeX\ can be made to report the calculations of the
paragraph mechanism in the log file. Some implementations of \TeX\
may have this option disabled to make \TeX\ run faster.

%%\spoint Three passes
%\subsection{Three passes}
%\spoint Three passes
\subsection{Three passes}

%First an attempt is made to split the paragraph into lines
%without hyphenating, that is, without inserting discretionary
%hyphens. This attempt succeeds if none of the
%lines has a badness exceeding \csidx{pretolerance}.
First an attempt is made to split the paragraph into lines
without hyphenating, that is, without inserting discretionary
hyphens. This attempt succeeds if none of the
lines has a badness exceeding \csidx{pretolerance}.

%Otherwise, a second pass is made, inserting discretionaries
%and using \csidx{tolerance}.
%\handbreak If \cs{pretolerance} is negative, the first pass is omitted.
Otherwise, a second pass is made, inserting discretionaries
and using \csidx{tolerance}.
\handbreak If \cs{pretolerance} is negative, the first pass is omitted.

%\TeX\ can be made to make a third pass if the first and
%second pass fail.
%If \csidx{emergencystretch} is a positive dimension,
%\TeX\ will assume this much extra stretchability
%in each line when badness and demerits are calculated.
%Thus solutions that only slightly exceeded the given
%tolerances will now become feasible.
%However, no glue of size \cs{emergencystretch} is
%actually present, so underfull box messages
%may still occur.
\TeX\ can be made to make a third pass if the first and
second pass fail.
If \csidx{emergencystretch} is a positive dimension,
\TeX\ will assume this much extra stretchability
in each line when badness and demerits are calculated.
Thus solutions that only slightly exceeded the given
tolerances will now become feasible.
However, no glue of size \cs{emergencystretch} is
actually present, so underfull box messages
may still occur.

%%\spoint Tolerance values
%\subsection{Tolerance values}
%\spoint Tolerance values
\subsection{Tolerance values}

%How much
%trouble \TeX\ will have typesetting a piece of text
%depends partly on the tolerance value.
%Therefore it is sensible to have some idea of
%what badness values mean in visual terms.
How much
trouble \TeX\ will have typesetting a piece of text
depends partly on the tolerance value.
Therefore it is sensible to have some idea of
what badness values mean in visual terms.

%For lines that are stretched, the badness is
%100 times the cube of the stretch ratio.
%A~badness of 800 thus means that the stretch ratio
%is~2.
%If the space is,
%\alt
%as in the ten-point Computer Modern Font,
%\begin{verbatim}
%3.33pt plus 1.67pt minus 1.11pt
%\end{verbatim}
%a badness of 800 means that spaces have been stretched to
%\begin{disp} \n{3.33pt}${}+2\times{}$\n{1.67pt}${}={}$\n{6.66pt}\end{disp}
%that is, to exactly double their natural size.
%It is up to you to decide whether this is too large.
For lines that are stretched, the badness is
100 times the cube of the stretch ratio.
A~badness of 800 thus means that the stretch ratio
is~2.
If the space is,
\alt
as in the ten-point Computer Modern Font,
\begin{verbatim}
3.33pt plus 1.67pt minus 1.11pt
\end{verbatim}
a badness of 800 means that spaces have been stretched to
\begin{disp} \n{3.33pt}${}+2\times{}$\n{1.67pt}${}={}$\n{6.66pt}\end{disp}
that is, to exactly double their natural size.
It is up to you to decide whether this is too large.

%%\point Discretionaries
%\section{Discretionaries}
%\point Discretionaries
\section{Discretionaries}

%A \indexterm{discretionary item} \verb-\discretionary{..}{..}{..}-
%\cstoidx discretionary\par
%marks a place where a word can be broken.
%Each of the three arguments is a \gr{general text}
%(see Chapter~\ref{gramm}):
%they are, in sequence,
%\begin{itemize} \item the {\em pre-break\/} text, which is appended
%to the part of the word before the break,
%\item the {\em post-break\/} text, which is prepended to the part
%of the word after the break, and
%\item the {\em no-break\/} text, which is used if the word
%is not broken at the discretionary item.\end{itemize}
%For example: \verb>ab\discretionary{g}{h}{cd}ef>
%is the word \hbox{\n{abcdef}}, but it can be hyphenated
%\alt
%with \n{abg} before the break and \n{hef} after.
%Note that there is no automatic hyphen character.
A \indexterm{discretionary item} \verb-\discretionary{..}{..}{..}-
\cstoidx discretionary\par
marks a place where a word can be broken.
Each of the three arguments is a \gr{general text}
(see Chapter~\ref{gramm}):
they are, in sequence,
\begin{itemize} \item the {\em pre-break\/} text, which is appended
to the part of the word before the break,
\item the {\em post-break\/} text, which is prepended to the part
of the word after the break, and
\item the {\em no-break\/} text, which is used if the word
is not broken at the discretionary item.\end{itemize}
For example: \verb>ab\discretionary{g}{h}{cd}ef>
is the word \hbox{\n{abcdef}}, but it can be hyphenated
\alt
with \n{abg} before the break and \n{hef} after.
Note that there is no automatic hyphen character.

%All three texts may contain any sorts of tokens,
%but any primitive commands and macros
%should expand to boxes, kerns, and characters.
All three texts may contain any sorts of tokens,
but any primitive commands and macros
should expand to boxes, kerns, and characters.

%%\spoint Hyphens and discretionaries
%\subsection{Hyphens and discretionaries}
%\spoint Hyphens and discretionaries
\subsection{Hyphens and discretionaries}

%Internally, \TeX\ inserts the equivalent of
%\cstoidx hyphenchar\par
%\begin{verbatim}
%\discretionary{\char\hyphenchar\font}{}{}
%\end{verbatim}
%at every place where a word can be broken. This causes a \indextermbus{hyphen}{character} to be placed before any break.
%No
%such discretionary is inserted if \verb>\hyphenchar\font>
%is not in the range 0--255, or if its position in the
%font is not filled.
%When a font is loaded, its \cs{hyphenchar} value
%is set to \csidx{defaulthyphenchar}. The \cs{hyphenchar}
%value can be changed after this.
Internally, \TeX\ inserts the equivalent of
\cstoidx hyphenchar\par
\begin{verbatim}
\discretionary{\char\hyphenchar\font}{}{}
\end{verbatim}
at every place where a word can be broken. This causes a \indextermbus{hyphen}{character} to be placed before any break.
No
such discretionary is inserted if \verb>\hyphenchar\font>
is not in the range 0--255, or if its position in the
font is not filled.
When a font is loaded, its \cs{hyphenchar} value
is set to \csidx{defaulthyphenchar}. The \cs{hyphenchar}
value can be changed after this.

%In plain \TeX\ the \cs{defaulthyphenchar} has the value~\verb>`\->, so
%for all fonts character~45 (the \ascii\ hyphen character)
%is the hyphen sign, unless
%it is specified otherwise.
In plain \TeX\ the \cs{defaulthyphenchar} has the value~\verb>`\->, so
for all fonts character~45 (the \ascii\ hyphen character)
is the hyphen sign, unless
it is specified otherwise.

%The primitive command \verb|\-| (called a `discretionary hyphen')
%\csterm -\par\indexterm{discretionary hyphen}
%is equivalent to the above
%\handbreak
%\verb|\discretionary{\char\hyphenchar\font}{}{}|.
%Breaking at such a discretionary, whether inserted implicitly
%by \TeX\ or explicitly by the user, has
%a cost of \csidx{hyphenpenalty}.
The primitive command \verb|\-| (called a `discretionary hyphen')
\csterm -\par\indexterm{discretionary hyphen}
is equivalent to the above
\handbreak
\verb|\discretionary{\char\hyphenchar\font}{}{}|.
Breaking at such a discretionary, whether inserted implicitly
by \TeX\ or explicitly by the user, has
a cost of \csidx{hyphenpenalty}.


%In unrestricted horizontal mode an empty discretionary
%\cs{disc\-re\-tio\-na\-ry}\verb-{}{}{}-
%is automatically inserted after characters
%whose character code is the \cs{hyphenchar} value
%of the font, thus enabling hyphenation at that point.
%The penalty for breaking a line at
%such a discretionary with an empty pre-break text
%is \csidx{exhyphenpenalty}, that is, the `explicit hyphen' penalty.
In unrestricted horizontal mode an empty discretionary
\cs{disc\-re\-tio\-na\-ry}\verb-{}{}{}-
is automatically inserted after characters
whose character code is the \cs{hyphenchar} value
of the font, thus enabling hyphenation at that point.
The penalty for breaking a line at
such a discretionary with an empty pre-break text
is \csidx{exhyphenpenalty}, that is, the `explicit hyphen' penalty.

%If a word contains
%discretionary breaks, for instance
%because of  explicit hyphen characters,
%\TeX\ will not consider it for further hyphenation.
%People have solved the ensuing problems by tricks
%such as
%\howto Enable hyphenation of a word containing a hyphen\par
%\begin{verbatim}
%\def\={\penalty10000 \hskip0pt -\penalty0 \hskip0pt\relax}
%... integro\=differential equations...
%\end{verbatim}
%The skips before and after the hyphen lead \TeX\ into
%treating the first and second half of the
%compound expression as separate words; the penalty
%before the first skip inhibits breaking before the hyphen.
If a word contains
discretionary breaks, for instance
because of  explicit hyphen characters,
\TeX\ will not consider it for further hyphenation.
People have solved the ensuing problems by tricks
such as
\howto Enable hyphenation of a word containing a hyphen\par
\begin{verbatim}
\def\={\penalty10000 \hskip0pt -\penalty0 \hskip0pt\relax}
... integro\=differential equations...
\end{verbatim}
The skips before and after the hyphen lead \TeX\ into
treating the first and second half of the
compound expression as separate words; the penalty
before the first skip inhibits breaking before the hyphen.

%%\spoint Examples of discretionaries
%\subsection{Examples of discretionaries}
%\spoint Examples of discretionaries
\subsection{Examples of discretionaries}

%\emph{Languages}\index{languages} such as German or Dutch have words
%that change spelling when hyphenated (German: `\hbox{backen}' becomes
%`\hbox{bak-ken}'; Dutch: `\hbox{autootje}' becomes
%`\hbox{auto-tje}'). This problem can be solved with \TeX's
%discretionaries.
\emph{Languages}\index{languages} such as German or Dutch have words
that change spelling when hyphenated (German: `\hbox{backen}' becomes
`\hbox{bak-ken}'; Dutch: `\hbox{autootje}' becomes
`\hbox{auto-tje}'). This problem can be solved with \TeX's
discretionaries.

%For instance, for German (this is inspired by~\cite{Partl}):
%\begin{verbatim}
%\catcode`\"=\active
%\def"#1{\ifx#1k\discretionary{k-}{k}{ck}\fi}
%\end{verbatim}
%which enables the user to write \verb>ba"ken>.
For instance, for German (this is inspired by~\cite{Partl}):
\begin{verbatim}
\catcode`\"=\active
\def"#1{\ifx#1k\discretionary{k-}{k}{ck}\fi}
\end{verbatim}
which enables the user to write \verb>ba"ken>.

%In Dutch there is a further problem which allows a nice
%systematic solution. Umlaut characters (`trema' is the
%Dutch term) should often
%disappear in a break, for instance `\hbox{na"apen}'
%hyphenates as `\hbox{na-apen}', and `\hbox{onbe"invloedbaar}'
%hyphenates as `\hbox{onbe-invloedbaar}'. A solution
%(inspired by~\cite{Babel}) is
%\begin{verbatim}
%\catcode`\"=\active
%\def"#1{\ifx#1i\discretionary{-}{i}{\"\i}%
%        \else  \discretionary{-}{#1}{\"#1}\fi}
%\end{verbatim}
%which enables the user to type \verb>na"apen> and
%\verb>onbe"invloedbaar>.
In Dutch there is a further problem which allows a nice
systematic solution. Umlaut characters (`trema' is the
Dutch term) should often
disappear in a break, for instance `\hbox{na"apen}'
hyphenates as `\hbox{na-apen}', and `\hbox{onbe"invloedbaar}'
hyphenates as `\hbox{onbe-invloedbaar}'. A solution
(inspired by~\cite{Babel}) is
\begin{verbatim}
\catcode`\"=\active
\def"#1{\ifx#1i\discretionary{-}{i}{\"\i}%
        \else  \discretionary{-}{#1}{\"#1}\fi}
\end{verbatim}
which enables the user to type \verb>na"apen> and
\verb>onbe"invloedbaar>.

%%\point Hyphenation
%\section{Hyphenation}
%\point Hyphenation
\section{Hyphenation}

%\TeX's \indexterm{hyphenation}
%algorithm uses a list of patterns to
%determine at what places a word that is a candidate for
%hyphenation can be broken.
%Those aspects of hyphenation connected with these
%patterns are
%treated in appendix~H of \TeXbook;
%the method of generating hyphenation patterns automatically
%is described in~\cite{Liang}. People have been known
%to generate lists of patterns by hand;
%see for instance~\cite{Vas:add}. Such hand-generated lists
%may be superior to automatically generated lists.
\TeX's \indexterm{hyphenation}
algorithm uses a list of patterns to
determine at what places a word that is a candidate for
hyphenation can be broken.
Those aspects of hyphenation connected with these
patterns are
treated in appendix~H of \TeXbook;
the method of generating hyphenation patterns automatically
is described in~\cite{Liang}. People have been known
to generate lists of patterns by hand;
see for instance~\cite{Vas:add}. Such hand-generated lists
may be superior to automatically generated lists.

%Here it will mainly be described how \TeX\ declares a word to
%be a candidate for hyphenation. The  problem here is
%how to cope with punctuation and things such as quotation marks
%that can be attached to a word. Also, {\em implicit kerns\/},
%that is, kerns inserted because of font information,
%must  be handled properly.
Here it will mainly be described how \TeX\ declares a word to
be a candidate for hyphenation. The  problem here is
how to cope with punctuation and things such as quotation marks
that can be attached to a word. Also, {\em implicit kerns\/},
that is, kerns inserted because of font information,
must  be handled properly.

%%\spoint Start of a word
%\subsection{Start of a word}
%\spoint Start of a word
\subsection{Start of a word}

%\TeX\ starts at glue items (if they are not in math mode)
%looking for a {\em starting letter\/} of a word:
%a character with non-zero \cs{lccode}, or a ligature starting
%\mdqon
%with such a character (upper/""lowercase codes are explained
%\mdqoff
%on page~\pageref{uc/lc}).
%Looking for this starting letter,
%\TeX\ bypasses any implicit kerns, and
%characters with zero \cs{lccode} (this includes,
%for instance, punctuation and quotation marks),
%or ligatures starting with
%such a character.
\TeX\ starts at glue items (if they are not in math mode)
looking for a {\em starting letter\/} of a word:
a character with non-zero \cs{lccode}, or a ligature starting
\mdqon
with such a character (upper/""lowercase codes are explained
\mdqoff
on page~\pageref{uc/lc}).
Looking for this starting letter,
\TeX\ bypasses any implicit kerns, and
characters with zero \cs{lccode} (this includes,
for instance, punctuation and quotation marks),
or ligatures starting with
such a character.

%If no suitable starting letter turns up, that is, if
%something is found that is not a character or ligature,
%\TeX\ skips to the next glue, and starts this algorithm anew.
%Otherwise a trial word is collected consisting of
%all following characters with non-zero \cs{lccode}
%from the same font as the starting letter, or ligatures consisting
%completely of such characters. Implicit kerns are allowed
%between the characters and ligatures.
If no suitable starting letter turns up, that is, if
something is found that is not a character or ligature,
\TeX\ skips to the next glue, and starts this algorithm anew.
Otherwise a trial word is collected consisting of
all following characters with non-zero \cs{lccode}
from the same font as the starting letter, or ligatures consisting
completely of such characters. Implicit kerns are allowed
between the characters and ligatures.

%If the starting letter is from a font for which the value
%of \cs{hyphenchar} is invalid, or for which this character
%does not exist, hyphenation is abandoned for this word.
%If the starting letter is an uppercase letter (that is,
%it is not equal to its own \cs{lccode}), \TeX\ will
%abandon hyphenation unless \csidx{uchyph} is positive.
%The default value for this parameter is~1  in
%plain \TeX,
%implying that capitalized words are subject to hyphenation.
If the starting letter is from a font for which the value
of \cs{hyphenchar} is invalid, or for which this character
does not exist, hyphenation is abandoned for this word.
If the starting letter is an uppercase letter (that is,
it is not equal to its own \cs{lccode}), \TeX\ will
abandon hyphenation unless \csidx{uchyph} is positive.
The default value for this parameter is~1  in
plain \TeX,
implying that capitalized words are subject to hyphenation.

%%\spoint End of a word
%\subsection{End of a word}
%\spoint End of a word
\subsection{End of a word}


%Following the trial word can be characters (from another
%font, or with zero \cs{lccode}), ligatures or implicit kerns.
%After these items, if any, must follow
%\begin{itemize}\item glue or an explicit kern,
%\item a penalty,
%\item a whatsit, or
%\item a \cs{mark}, \cs{insert}, or \cs{vadjust} item.
%\end{itemize}
%In particular, the word will not be hyphenated if it is
%followed by a
%\begin{itemize}\item box, \item rule, \item math
%formula, or \item discretionary item.\end{itemize}
Following the trial word can be characters (from another
font, or with zero \cs{lccode}), ligatures or implicit kerns.
After these items, if any, must follow
\begin{itemize}\item glue or an explicit kern,
\item a penalty,
\item a whatsit, or
\item a \cs{mark}, \cs{insert}, or \cs{vadjust} item.
\end{itemize}
In particular, the word will not be hyphenated if it is
followed by a
\begin{itemize}\item box, \item rule, \item math
formula, or \item discretionary item.\end{itemize}

%Since discretionaries are inserted after the \cs{hyphenchar}
%of the font, occurrence of this character inhibits further
%hyphenation. Also, placement of accents is implemented using
%explicit kerns (see Chapter~\ref{char}), so any \cs{accent}
%command is considered to be the end of a word, and inhibits
%hyphenation of the word.
Since discretionaries are inserted after the \cs{hyphenchar}
of the font, occurrence of this character inhibits further
hyphenation. Also, placement of accents is implemented using
explicit kerns (see Chapter~\ref{char}), so any \cs{accent}
command is considered to be the end of a word, and inhibits
hyphenation of the word.

%%\spoint \TeX2 versus \TeX3
%\subsection{\TeX2 versus \TeX3}
%\spoint \TeX2 versus \TeX3
\subsection{\TeX2 versus \TeX3}

%There is a noticeable difference in the treatment of
%\thecstoidxsub{TeX}{version 3}
%hyphenated fragments between \TeX2 and \TeX3.
%\TeX2 insists that the part before the break should be
%at least two characters, and the part after the break three
%characters, long.
%Typographically this is a sound decision: this way
%there are no two-character pieces of a word stranded at the
%end or beginning of the line. Both before and after the break
%there are at least three characters.
There is a noticeable difference in the treatment of
\thecstoidxsub{TeX}{version 3}
hyphenated fragments between \TeX2 and \TeX3.
\TeX2 insists that the part before the break should be
at least two characters, and the part after the break three
characters, long.
Typographically this is a sound decision: this way
there are no two-character pieces of a word stranded at the
end or beginning of the line. Both before and after the break
there are at least three characters.

%In \TeX3 two integer parameters have been introduced to control
%the length of these fragments:
%\csidx{lefthyphenmin} and \csidx{righthyphenmin}. These are
%set to 2 and~3 respectively in the plain format for \TeX3.
%If the sum of these two is 63 or more, all hyphenation is
%suppressed.
In \TeX3 two integer parameters have been introduced to control
the length of these fragments:
\csidx{lefthyphenmin} and \csidx{righthyphenmin}. These are
set to 2 and~3 respectively in the plain format for \TeX3.
If the sum of these two is 63 or more, all hyphenation is
suppressed.

%Another addition in \TeX3,
%the possibility to have several sets of hyphenation patterns,
%is treated below.
Another addition in \TeX3,
the possibility to have several sets of hyphenation patterns,
is treated below.

%%\spoint Patterns and exceptions
%\subsection{Patterns and exceptions}
%\spoint Patterns and exceptions
\subsection{Patterns and exceptions}

%The statements
%\begin{disp}\cs{patterns}\gr{general text}\nl
%\csidx{hyphenation}\gr{general text}\end{disp}
%are \gr{hyphenation assignment}s, which are
%\gr{global assignment}s.
%The \csidx{patterns} command, which specifies a list
%of hyphenation patterns, is allowed only in \IniTeX\
%(see Chapter~\ref{TeXcomm}),
%and all patterns must be specified before the first
%paragraph is typeset.
The statements
\begin{disp}\cs{patterns}\gr{general text}\nl
\csidx{hyphenation}\gr{general text}\end{disp}
are \gr{hyphenation assignment}s, which are
\gr{global assignment}s.
The \csidx{patterns} command, which specifies a list
of hyphenation patterns, is allowed only in \IniTeX\
(see Chapter~\ref{TeXcomm}),
and all patterns must be specified before the first
paragraph is typeset.

%Hyphenation exceptions can be specified at any time
%\howto Specify exceptional hyphenations\par
%with statements such as
%\begin{verbatim}
%\hyphenation{oxy-mo-ron gar-goyle}
%\end{verbatim}
%which specify locations where a word may be hyphenated.
%Subsequent \cs{hyphenation} statements are cumulative.
Hyphenation exceptions can be specified at any time
\howto Specify exceptional hyphenations\par
with statements such as
\begin{verbatim}
\hyphenation{oxy-mo-ron gar-goyle}
\end{verbatim}
which specify locations where a word may be hyphenated.
Subsequent \cs{hyphenation} statements are cumulative.

%In \TeX3 these statements are taken to hold for the
%language that is the current value of the \cs{language}
%parameter.
In \TeX3 these statements are taken to hold for the
language that is the current value of the \cs{language}
parameter.

%%\point Switching hyphenation patterns
%\section{Switching hyphenation patterns}
%\point Switching hyphenation patterns
\section{Switching hyphenation patterns}

%When typesetting paragraphs, \TeX\ (version~3) can use several
%\alt
%sets of patterns and hyphenation exceptions, for at most 256
%\emph{languages}\index{language}.
When typesetting paragraphs, \TeX\ (version~3) can use several
\alt
sets of patterns and hyphenation exceptions, for at most 256
\emph{languages}\index{language}.

%If a \cs{patterns} or \cs{hyphenation}
%command is given (see above), \TeX\ stores the patterns or exceptions
%under the current value of the \csidx{language} parameter.
%The \cs{patterns} command is only allowed in \IniTeX, and
%patterns must be specified before any typesetting is done.
%Hyphenation exceptions, however, can
%be specified cumulatively, and not only in \IniTeX.
If a \cs{patterns} or \cs{hyphenation}
command is given (see above), \TeX\ stores the patterns or exceptions
under the current value of the \csidx{language} parameter.
The \cs{patterns} command is only allowed in \IniTeX, and
patterns must be specified before any typesetting is done.
Hyphenation exceptions, however, can
be specified cumulatively, and not only in \IniTeX.

%In addition to the \cs{language} parameter, which can be set by the
%user, \TeX\ has internally a \indextermbus{current}{language}. This is
%set to zero at the start of every paragraph.  For every character that
%is added to a paragraph the current language is compared with the
%value of \cs{language}, and if they differ a whatsit element is added
%to the horizontal list, resetting the current language to the value of
%\cs{language}.
In addition to the \cs{language} parameter, which can be set by the
user, \TeX\ has internally a \indextermbus{current}{language}. This is
set to zero at the start of every paragraph.  For every character that
is added to a paragraph the current language is compared with the
value of \cs{language}, and if they differ a whatsit element is added
to the horizontal list, resetting the current language to the value of
\cs{language}.

%At the start of a paragraph, this whatsit is inserted
%\altt
%after the \cs{everypar} tokens, but \cs{lastbox}
%can still access the indentation box.
At the start of a paragraph, this whatsit is inserted
\altt
after the \cs{everypar} tokens, but \cs{lastbox}
can still access the indentation box.

%As an example, suppose that a format has been created such that
%language~0 is English, and language~1 is Dutch. English hyphenations
%will then be used if the user does not specify otherwise;
%if a job starts with
%\begin{verbatim}
%\language=1
%\end{verbatim}
%the whole document
%will be set using Dutch hyphenations, because \TeX\ will insert
%a command changing the current language at the start of
%every paragraph. For example:
%\begin{verbatim}
%\language=1
%T...
%\end{verbatim}
%gives
%\begin{verbatim}
%.\hbox(0.0+0.0)x20.0           % indentation
%.\setlanguage1 (hyphenmin 2,3) % language whatsit
%.\tenrm T                      % start of text
%\end{verbatim}
As an example, suppose that a format has been created such that
language~0 is English, and language~1 is Dutch. English hyphenations
will then be used if the user does not specify otherwise;
if a job starts with
\begin{verbatim}
\language=1
\end{verbatim}
the whole document
will be set using Dutch hyphenations, because \TeX\ will insert
a command changing the current language at the start of
every paragraph. For example:
\begin{verbatim}
\language=1
T...
\end{verbatim}
gives
\begin{verbatim}
.\hbox(0.0+0.0)x20.0           % indentation
.\setlanguage1 (hyphenmin 2,3) % language whatsit
.\tenrm T                      % start of text
\end{verbatim}

%The whatsit can be inserted explicitly, without changing
%the value of \csidx{language}, by specifying
%\begin{disp}\cs{setlanguage}\gr{number}\end{disp}
%However, this will hardly ever be needed.
%One case where it may be necessary is when the contents of
%a horizontal box are unboxed to a paragraph: inside the box no
%whatsits are added automatically, since inside such a box
%no hyphenation can take place.
%See page~\pageref{wide:vbox} for another problem with text
%in horizontal boxes.
The whatsit can be inserted explicitly, without changing
the value of \csidx{language}, by specifying
\begin{disp}\cs{setlanguage}\gr{number}\end{disp}
However, this will hardly ever be needed.
One case where it may be necessary is when the contents of
a horizontal box are unboxed to a paragraph: inside the box no
whatsits are added automatically, since inside such a box
no hyphenation can take place.
See page~\pageref{wide:vbox} for another problem with text
in horizontal boxes.

%\endofchapter
\endofchapter

\end{document}

