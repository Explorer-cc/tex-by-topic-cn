% -*- coding: utf-8 -*-
\documentclass{book}

% -*- coding: utf-8 -*-

\usepackage[b5paper,text={5in,8in},centering]{geometry}
\usepackage{amsmath}
\usepackage{amssymb}
\usepackage[heading = false, scheme = plain, fontset = fandol]{ctex}
% \usepackage[CJKchecksingle]{xeCJK}
% \setmainfont[Mapping=tex-text]{TeX Gyre Schola}
\setmainfont[Mapping=tex-text]{TeX Gyre Pagella}
%\setsansfont{URW Gothic L Book}
%\setmonofont{Nimbus Mono L}
% \setCJKmainfont[BoldFont=FandolHei,ItalicFont=FandolKai]{FandolSong}
% \setCJKsansfont{FandolHei}
% \setCJKmonofont{FandolFang}
\xeCJKsetup{PunctStyle = kaiming}

\linespread{1.25}
\setlength{\parindent}{2em}
\setlength{\parskip}{0.5ex}
\usepackage{indentfirst}

\usepackage{xcolor}
\definecolor{myblue}{rgb}{0,0.2,0.6}

\usepackage{titlesec}
\titleformat{\chapter}
    {\normalfont\Huge\sffamily\color{myblue}}
    {第\thechapter 章}
    {1em}
    {}
%\titlespacing{\chapter}{0pt}{50pt}{40pt}
\titleformat{\section}
    {\normalfont\Large\sffamily\color{myblue}}
    {\thesection}
    {1em}
    {}
%\titlespacing{\section}{0pt}{3.5ex plus 1ex minus .2ex}{2.3ex plus .2ex}
\titleformat{\subsection}
    {\normalfont\large\sffamily\color{myblue}}
    {\thesubsection}
    {1em}
    {}
%\titlespacing{\subsection}{0pt}{3.25ex plus 1ex minus .2ex}{1.5ex plus .2ex}
%
\newpagestyle{special}[\small\sffamily]{
  \headrule
  \sethead[\usepage][][\chaptertitle]
  {\chaptertitle}{}{\usepage}}
\newpagestyle{main}[\small\sffamily]{
  \headrule
  \sethead[\usepage][][第\thechapter 章\quad\chaptertitle]
  {\thesection\quad\sectiontitle}{}{\usepage}}

\usepackage{titletoc}
%\setcounter{tocdepth}{1}
%\titlecontents{标题层次}[左间距]{上间距和整体格式}{标题序号}{标题内容}{指引线和页码}[下间距]
\titlecontents{chapter}[1.5em]{\vspace{.5em}\bfseries\sffamily}{\color{myblue}\contentslabel{1.5em}}{}
    {\titlerule*[20pt]{$\cdot$}\contentspage}[]
\titlecontents{section}[4.5em]{\sffamily}{\color{myblue}\contentslabel{3em}}{}
    {\titlerule*[20pt]{$\cdot$}\contentspage}[]
%\titlecontents{subsection}[8.5em]{\sffamily}{\contentslabel{4em}}{}
%    {\titlerule*[20pt]{$\cdot$}\contentspage}

\usepackage{enumitem}
\setlist{topsep=2pt,itemsep=2pt,parsep=1pt,leftmargin=\parindent}

\usepackage{fancyvrb}
\DefineVerbatimEnvironment{verbatim}{Verbatim}
  {xleftmargin=2em,baselinestretch=1,formatcom=\color{teal}\upshape}

\usepackage{etoolbox}
\makeatletter
\preto{\FV@ListVSpace}{\topsep=2pt \partopsep=0pt }
\makeatother

\usepackage[colorlinks,plainpages,pagebackref]{hyperref}
\hypersetup{
   pdfstartview={FitH},
   citecolor=teal,
   linkcolor=myblue,
   urlcolor=black,
   bookmarksnumbered
}
\makeatletter
\def\UrlAlphabet{%
      \do\a\do\b\do\c\do\d\do\e\do\f\do\g\do\h\do\i\do\j%
      \do\k\do\l\do\m\do\n\do\o\do\p\do\q\do\r\do\s\do\t%
      \do\u\do\v\do\w\do\x\do\y\do\z\do\A\do\B\do\C\do\D%
      \do\E\do\F\do\G\do\H\do\I\do\J\do\K\do\L\do\M\do\N%
      \do\O\do\P\do\Q\do\R\do\S\do\T\do\U\do\V\do\W\do\X%
      \do\Y\do\Z}
\def\UrlDigits{\do\1\do\2\do\3\do\4\do\5\do\6\do\7\do\8\do\9\do\0}
\g@addto@macro{\UrlBreaks}{\UrlOrds}
\g@addto@macro{\UrlBreaks}{\UrlAlphabet}
\g@addto@macro{\UrlBreaks}{\UrlDigits}
\makeatother

\usepackage{comment,makeidx,multicol}

%\usepackage{german}
%% german
%\righthyphenmin=3
%\mdqoff
%\captionsenglish
\usepackage[english]{babel}
{\catcode`"=13 \gdef"#1{\ifx#1"\discretionary{}{}{}\fi\relax}}
\def\mdqon{\catcode`"=13\relax}
\def\mdqoff{\catcode`"=12\relax}
\makeindex
\hyphenation{ex-em-pli-fies}

\newdimen\tempdima \newdimen\tempdimb

% these are fine
\def\handbreak{\\ \message{^^JManual break!!!!^^J}}
\def\nl{\protect\\}\def\n#1{{\tt #1}}
\protected\def\cs#1{\texttt{\textbackslash#1}}
\pdfstringdefDisableCommands{\def\cs#1{\textbackslash#1}}
\let\csc\cs
\def\lb{{\tt\char`\{}}\def\rb{{\tt\char`\}}}
\def\gr#1{\texorpdfstring{$\langle$#1$\rangle$}{<#1>}} %\def\gr#1{$\langle$#1$\rangle$}
\def\marg#1{{\tt \{}#1{\tt \}}}
\def\oarg#1{{\tt [}#1{\tt }}
\def\key#1{{\tt#1}}
\def\alt{}\def\altt{}%this way in manstijl
\def\ldash{\unskip\ ——\nobreak\ \ignorespaces}
\def\rdash{\unskip\nobreak\ ——\ \ignorespaces}
% check these
\def\hex{{\tt"}}
\def\ascii{{\sc ascii}}
\def\ebcdic{{\sc ebcdic}}
\def\IniTeX{Ini\TeX}\def\LamsTeX{LAMS\TeX}\def\VirTeX{Vir\TeX}
\def\AmsTeX{Ams\TeX}
\def\TeXbook{the \TeX\ book}\def\web{{\sc web}}
% needs major thinking
\newenvironment{myquote}{\list{}{%
    \topsep=2pt \partopsep=0pt%
    \leftmargin=\parindent \rightmargin=\parindent
    }\item[]}{\endlist}
\newenvironment{disp}{\begin{myquote}}{\end{myquote}}
\newenvironment{Disp}{\begin{myquote}}{\end{myquote}}
\newenvironment{tdisp}{\begin{myquote}}{\end{myquote}}
\newenvironment{example}{\begin{myquote}\noindent\itshape 例子:}{\end{myquote}}
\newenvironment{inventory}{\begin{description}\raggedright}{\end{description}}
\newenvironment{glossinventory}{\begin{description}}{\end{description}}
\def\gram#1{\gr{#1}}%???
\def\meta{\gr}% alias
%
% index
%
\def\indexterm#1{\emph{#1}\index{#1}}
\def\indextermsub#1#2{\emph{#1 #2}\index{#1!#2}}
\def\indextermbus#1#2{\emph{#1 #2}\index{#2!#1}}
\def\cindextermsub#1#2{\emph{#1#2}\index{#1!#1#2}}
\def\cindextermbus#1#2{\emph{#1#2}\index{#2!#1#2}}
\def\term#1\par{\index{#1}}
\def\howto#1\par{}
\def\cstoidx#1\par{\index{#1@\cs{#1}@}}
\def\thecstoidx#1\par{\index{#1@\protect\csname #1\endcsname}}
\def\thecstoidxsub#1#2{\index{#1, #2@\protect\csname #1\endcsname, #2}\ignorespaces}
\def\csterm#1\par{\cstoidx #1\par\cs{#1}}
\def\csidx#1{\cstoidx #1\par\cs{#1}}

\def\tmc{\tracingmacros=2 \tracingcommands\tracingmacros}

%%%%%%%%%%%%%%%%%%%
\makeatletter
\def\snugbox{\hbox\bgroup\setbox\z@\vbox\bgroup
    \leftskip\z@
    \bgroup\aftergroup\make@snug
    \let\next=}
\def\make@snug{\par\sn@gify\egroup \box\z@\egroup}
\def\sn@gify
   {\skip\z@=\lastskip \unskip
    \advance\skip\z@\lastskip \unskip
    \unpenalty
    \setbox\z@\lastbox
    \ifvoid\z@ \nointerlineskip \else {\sn@gify} \fi
    \hbox{\unhbox\z@}\nointerlineskip
    \vskip\skip\z@
    }

\newdimen\fbh \fbh=60pt % dimension for easy scaling:
\newdimen\fbw \fbw=60pt % height and width of character box

\newdimen\dh \newdimen\dw % height and width of current character box
\newdimen\lh % height of previous character box
\newdimen\lw \lw=.4pt % line weight, instead of default .4pt

\def\hdotfill{\noindent
    \leaders\hbox{\vrule width 1pt height\lw
                  \kern4pt
                  \vrule width.5pt height\lw}\hfill\hbox{}
    \par}
\def\hlinefill{\noindent
    \leaders\hbox{\vrule width 5.5pt height\lw         }\hfill\hbox{}
    \par}
\def\stippel{$\qquad\qquad\qquad\qquad$}
\makeatother
%%%%%%%%%%%%%%%%%%%

%\def\SansSerif{\Typeface:macHelvetica }
%\def\SerifFont{\Typeface:macTimes }
%\def\SansSerif{\Typeface:bsGillSans }
%\def\SerifFont{\Typeface:bsBaskerville }
\let\SansSerif\relax \def\italic{\it}
\let\SerifFont\relax \def\MainFont{\rm}
\let\SansSerif\relax
\let\SerifFont\relax
\let\PopIndentLevel\relax \let\PushIndentLevel\relax
\let\ToVerso\relax \let\ToRecto\relax

%\def\stop@command@suffix{stop}
%\let\PopListLevel\PopIndentLevel
%\let\FlushRight\relax
%\let\flushright\FlushRight
%\let\SetListIndent\LevelIndent
%\def\awp{\ifhmode\vadjust{\penalty-10000 }\else
%    \penalty-10000 \fi}
\let\awp\relax
\let\PopIndentLevel\relax \let\PopListLevel\relax

\showboxdepth=-1

%\input figs
\def\endofchapter{\vfill\noindent}

\newcommand{\liamfnote}[1]{\protect\footnote{译注(Liam0205):#1}}
\newcommand{\cstate}[1]{状态 \textit{#1}}

\setcounter{chapter}{7}

\begin{document}

%\chapter{Dimensions and Glue}\label{glue}
\chapter{Dimensions and Glue}\label{glue}

%In \TeX\ vertical and horizontal white space
%can have a possibility to adjust itself through `stretching' or
%`shrinking'. An~adjustable white space is called \indexterm{glue}.
%This chapter treats all technical concepts related to
%dimensions and glue, and it explains how the badness of stretching or shrinking
%a  certain amount is calculated.
In \TeX\ vertical and horizontal white space
can have a possibility to adjust itself through `stretching' or
`shrinking'. An~adjustable white space is called \indexterm{glue}.
This chapter treats all technical concepts related to
dimensions and glue, and it explains how the badness of stretching or shrinking
a  certain amount is calculated.

%\label{cschap:dimen}\label{cschap:dimendef}\label{cschap:newdimen}\label{cschap:skip}\label{cschap:skipdef}\label{cschap:newskip}\label{cschap:hskip}\label{cschap:hfil}\label{cschap:hfilneg}\label{cschap:hfill}\label{cschap:hss}\label{cschap:vskip}\label{cschap:vfil}\label{cschap:vfill}\label{cschap:vfilneg}\label{cschap:vss}\label{cschap:kern}\label{cschap:lastkern}\label{cschap:lastskip}\label{cschap:unkern}\label{cschap:unskip}\label{cschap:removelastskip}\label{cschap:advance2}\label{cschap:multiply2}\label{cschap:divide2}
%\begin{inventory}
%\item [\cs{dimen}] 
%      Dimension register prefix.
\label{cschap:dimen}\label{cschap:dimendef}\label{cschap:newdimen}\label{cschap:skip}\label{cschap:skipdef}\label{cschap:newskip}\label{cschap:hskip}\label{cschap:hfil}\label{cschap:hfilneg}\label{cschap:hfill}\label{cschap:hss}\label{cschap:vskip}\label{cschap:vfil}\label{cschap:vfill}\label{cschap:vfilneg}\label{cschap:vss}\label{cschap:kern}\label{cschap:lastkern}\label{cschap:lastskip}\label{cschap:unkern}\label{cschap:unskip}\label{cschap:removelastskip}\label{cschap:advance2}\label{cschap:multiply2}\label{cschap:divide2}
\begin{inventory}
\item [\cs{dimen}] 
      Dimension register prefix.

%\item [\cs{dimendef}] 
%      Define a control sequence to be a synonym for
%      a~\cs{dimen} register.
\item [\cs{dimendef}] 
      Define a control sequence to be a synonym for
      a~\cs{dimen} register.

%\item [\cs{newdimen}] 
%      Allocate an unused dimen register. 
\item [\cs{newdimen}] 
      Allocate an unused dimen register. 

%\item [\cs{skip}] 
%      Skip register prefix.
\item [\cs{skip}] 
      Skip register prefix.

%\item [\cs{skipdef}] 
%      Define a control sequence to be a synonym for
%      a~\cs{skip} register.
\item [\cs{skipdef}] 
      Define a control sequence to be a synonym for
      a~\cs{skip} register.

%\item [\cs{newskip}]
%      Allocate an unused skip register.
\item [\cs{newskip}]
      Allocate an unused skip register.

%\item [\cs{ifdim}] 
%      Compare two dimensions. 
\item [\cs{ifdim}] 
      Compare two dimensions. 

%\item [\cs{hskip}]  
%      Insert in horizontal mode a glue item.
\item [\cs{hskip}]  
      Insert in horizontal mode a glue item.

%\item [\csidx{hfil}] 
%      Equivalent to 
%      \verb-\hskip 0cm plus 1fil-.
\item [\csidx{hfil}] 
      Equivalent to 
      \verb-\hskip 0cm plus 1fil-.

%\item [\csidx{hfilneg}] 
%      Equivalent to 
%      \verb-\hskip 0cm minus 1fil-.
\item [\csidx{hfilneg}] 
      Equivalent to 
      \verb-\hskip 0cm minus 1fil-.

%\item [\csidx{hfill}] 
%      Equivalent to 
%      \verb-\hskip 0cm plus 1fill-.
\item [\csidx{hfill}] 
      Equivalent to 
      \verb-\hskip 0cm plus 1fill-.

%\item [\csidx{hss}] 
%      Equivalent to 
%      \verb-\hskip 0cm plus 1fil minus 1fil-.
\item [\csidx{hss}] 
      Equivalent to 
      \verb-\hskip 0cm plus 1fil minus 1fil-.

%\item [\cs{vskip}]  
%      Insert in vertical mode a glue item.
\item [\cs{vskip}]  
      Insert in vertical mode a glue item.

%\item [\csidx{vfil}] 
%      Equivalent to 
%      \verb-\vskip 0cm plus 1fil-.
\item [\csidx{vfil}] 
      Equivalent to 
      \verb-\vskip 0cm plus 1fil-.

%\item [\csidx{vfill}] 
%      Equivalent to 
%      \verb-\vskip 0cm plus 1fill-.
\item [\csidx{vfill}] 
      Equivalent to 
      \verb-\vskip 0cm plus 1fill-.

%\item [\csidx{vfilneg}] 
%      Equivalent to 
%      \verb-\vskip 0cm minus 1fil-.
\item [\csidx{vfilneg}] 
      Equivalent to 
      \verb-\vskip 0cm minus 1fil-.

%\item [\csidx{vss}] 
%      Equivalent to 
%      \verb-\vskip 0cm plus 1fil minus 1fil-.
\item [\csidx{vss}] 
      Equivalent to 
      \verb-\vskip 0cm plus 1fil minus 1fil-.

%\item [\cs{kern}]  
%      Add a kern item to the current horizontal or vertical list.
\item [\cs{kern}]  
      Add a kern item to the current horizontal or vertical list.

%\item [\cs{lastkern}] 
%      If the last item on the current list was a kern, the size of it.
\item [\cs{lastkern}] 
      If the last item on the current list was a kern, the size of it.

%\item [\cs{lastskip}] 
%      If the last item on the current list was a~glue, the size of it.
\item [\cs{lastskip}] 
      If the last item on the current list was a~glue, the size of it.

%\item [\cs{unkern}] 
%      If the last item of the current list was a~kern, remove it.
\item [\cs{unkern}] 
      If the last item of the current list was a~kern, remove it.

%\item [\cs{unskip}] 
%      If the last item of the current list was a~glue, remove it.
\item [\cs{unskip}] 
      If the last item of the current list was a~glue, remove it.

%\item [\cs{removelastskip}]
%      Macro to append the negative of the \cs{lastskip}.
\item [\cs{removelastskip}]
      Macro to append the negative of the \cs{lastskip}.

%\item [\cs{advance}] 
%      Arithmetic command to add to or subtract from
%      a~\gr{numeric variable}.
\item [\cs{advance}] 
      Arithmetic command to add to or subtract from
      a~\gr{numeric variable}.

%\item [\cs{multiply}] 
%      Arithmetic command to multiply a~\gr{numeric variable}.
\item [\cs{multiply}] 
      Arithmetic command to multiply a~\gr{numeric variable}.

%\item [\cs{divide}] 
%      Arithmetic command to divide a~\gr{numeric variable}.
\item [\cs{divide}] 
      Arithmetic command to divide a~\gr{numeric variable}.

%\end{inventory}
\end{inventory}



%%\point Definition of \gr{glue} and \gr{dimen}
%\section{Definition of \gr{glue} and \gr{dimen}}
%\point Definition of \gr{glue} and \gr{dimen}
\section{Definition of \gr{glue} and \gr{dimen}}

%This section gives
%the syntax of the quantities
%\gr{dimen} and \gr{glue}. 
%In the next section the practical aspects of glue are treated.
This section gives
the syntax of the quantities
\gr{dimen} and \gr{glue}. 
In the next section the practical aspects of glue are treated.

%Unfortunately the terminology for glue is slightly confusing.
%The syntactical quantity~\gr{glue} is a dimension (a distance) with
%\mdqon
%possibly a stretch and/""or shrink component.
%\mdqoff
%In order to add a glob of `glue' (a white space) to a list one has to
%let a \gr{glue} be preceded by commands such as \cs{vskip}.
Unfortunately the terminology for glue is slightly confusing.
The syntactical quantity~\gr{glue} is a dimension (a distance) with
\mdqon
possibly a stretch and/""or shrink component.
\mdqoff
In order to add a glob of `glue' (a white space) to a list one has to
let a \gr{glue} be preceded by commands such as \cs{vskip}.


%%\spoint Definition of dimensions
%\subsection{Definition of dimensions}
%\spoint Definition of dimensions
\subsection{Definition of dimensions}

%A~\gr{dimen} is what \TeX\ expects to see when
%it needs to indicate a dimension; it can be positive or negative.
%\begin{disp}\gr{dimen} $\longrightarrow$ \gr{optional signs}%
%     \gr{unsigned dimen}\end{disp}
%The unsigned part of a \gr{dimen} can be
%\begin{disp}\gr{unsigned dimen} $\longrightarrow$ \gr{normal dimen}
%     $|$ \gr{coerced dimen}\nl
%     \gr{normal dimen} $\longrightarrow$ \gr{internal dimen}
%     $|$ \gr{factor}\gr{unit of measure}\nl
%     \gr{coerced dimen} $\longrightarrow$ \gr{internal glue}
%     \end{disp}
%That is, we have the following three cases:
%\begin{itemize} \item an \gr{internal dimen}; this is
% any register or parameter of \TeX\ that has a \gr{dimen} value:
% \begin{disp}\PopIndentLevel\gr{internal dimen} $\longrightarrow$
%      \gr{dimen parameter}\nl
%      \indent $|$ \gr{special dimen} $|$ \cs{lastkern}\nl
%      \indent $|$ \gr{dimendef token} $|$ \cs{dimen}\gr{8-bit number}\nl
%      \indent $|$ \cs{fontdimen}\gr{number}\gr{font}\nl
%      \indent $|$ \gr{box dimension}\gr{8-bit number}\nl
%      \gr{dimen parameter} $\longrightarrow$ \cs{boxmaxdepth}\nl
%      \indent $|$ \cs{delimitershortfall} $|$ \cs{displayindent}\nl
%      \indent $|$ \cs{displaywidth} $|$ \cs{hangindent}\nl
%      \indent $|$ \cs{hfuzz} $|$ \cs{hoffset} $|$ \cs{hsize}\nl
%      \indent $|$ \cs{lineskiplimit} $|$ \cs{mathsurround}\nl
%      \indent $|$ \cs{maxdepth} $|$ \cs{nulldelimiterspace}\nl
%      \indent $|$ \cs{overfullrule} $|$ \cs{parindent}\nl
%      \indent $|$ \cs{predisplaysize} $|$ \cs{scriptspace}\nl
%      \indent $|$ \cs{splitmaxdepth} $|$ \cs{vfuzz}\nl
%      \indent $|$ \cs{voffset} $|$ \cs{vsize}
% \end{disp}
%\item  a dimension denotation, 
% consisting of \gr{factor}\gr{unit of measure},
% for example \verb>0.7\vsize>; or
%\item an \gr{internal glue} (see below) 
% coerced to a dimension by omitting
% the stretch and shrink components, for example \cs{parfillskip}.
%\end{itemize}
A~\gr{dimen} is what \TeX\ expects to see when
it needs to indicate a dimension; it can be positive or negative.
\begin{disp}\gr{dimen} $\longrightarrow$ \gr{optional signs}%
     \gr{unsigned dimen}\end{disp}
The unsigned part of a \gr{dimen} can be
\begin{disp}\gr{unsigned dimen} $\longrightarrow$ \gr{normal dimen}
     $|$ \gr{coerced dimen}\nl
     \gr{normal dimen} $\longrightarrow$ \gr{internal dimen}
     $|$ \gr{factor}\gr{unit of measure}\nl
     \gr{coerced dimen} $\longrightarrow$ \gr{internal glue}
     \end{disp}
That is, we have the following three cases:
\begin{itemize} \item an \gr{internal dimen}; this is
 any register or parameter of \TeX\ that has a \gr{dimen} value:
 \begin{disp}\PopIndentLevel\gr{internal dimen} $\longrightarrow$
      \gr{dimen parameter}\nl
      \indent $|$ \gr{special dimen} $|$ \cs{lastkern}\nl
      \indent $|$ \gr{dimendef token} $|$ \cs{dimen}\gr{8-bit number}\nl
      \indent $|$ \cs{fontdimen}\gr{number}\gr{font}\nl
      \indent $|$ \gr{box dimension}\gr{8-bit number}\nl
      \gr{dimen parameter} $\longrightarrow$ \cs{boxmaxdepth}\nl
      \indent $|$ \cs{delimitershortfall} $|$ \cs{displayindent}\nl
      \indent $|$ \cs{displaywidth} $|$ \cs{hangindent}\nl
      \indent $|$ \cs{hfuzz} $|$ \cs{hoffset} $|$ \cs{hsize}\nl
      \indent $|$ \cs{lineskiplimit} $|$ \cs{mathsurround}\nl
      \indent $|$ \cs{maxdepth} $|$ \cs{nulldelimiterspace}\nl
      \indent $|$ \cs{overfullrule} $|$ \cs{parindent}\nl
      \indent $|$ \cs{predisplaysize} $|$ \cs{scriptspace}\nl
      \indent $|$ \cs{splitmaxdepth} $|$ \cs{vfuzz}\nl
      \indent $|$ \cs{voffset} $|$ \cs{vsize}
 \end{disp}
\item  a dimension denotation, 
 consisting of \gr{factor}\gr{unit of measure},
 for example \verb>0.7\vsize>; or
\item an \gr{internal glue} (see below) 
 coerced to a dimension by omitting
 the stretch and shrink components, for example \cs{parfillskip}.
\end{itemize}

%A dimension denotation is a somewhat complicated entity:
%\begin{itemize} \item a \gr{factor} is an integer denotation,
% a decimal constant denotation (a number with an integral and
% a fractional part),
% or an \gr{internal integer}
% \begin{disp}\PopIndentLevel
%      \gr{factor} $\longrightarrow$ \gr{normal integer} 
%      $|$ \gr{decimal constant}\nl
%      \gr{normal integer} $\longrightarrow$ \gr{integer denotation}\nl
%      \indent $|$ \gr{internal integer}\nl
%      \gr{decimal constant} $\longrightarrow$ \n{.$_{12}$}
%      $|$ \n{,$_{12}$}\nl
%      \indent $|$ \gr{digit}\gr{decimal constant}\nl
%      \indent $|$ \gr{decimal constant}\gr{digit}
% \end{disp}
% An internal integer is a parameter that is `really' an
%\alt
% integer (for instance, \cs{count0}), and not coerced from a dimension or glue.
% See Chapter~\ref{number}
% for the definition of various kinds of integers.
%\item a \gr{unit of measure} can be 
% a \gr{physical unit}, that is, an ordinary unit such as~\n{cm} 
% (possibly preceded by \n{true}),
% an internal unit such as~\n{em}, but also an \gr{internal integer}
% (by conversion to scaled points),
% an \gr{internal dimen}, or an \gr{internal glue}.
% \begin{disp}\PopIndentLevel
%      \gr{unit of measure} $\longrightarrow$
%      \gr{optional spaces}\gr{internal unit}\nl
%      \indent $|$ 
%      \gr{optional \n{true}}\gr{physical unit}\gr{one optional space}\nl 
%      \gr{internal unit} $\longrightarrow$ 
%      \n{em}\gr{one optional space}\nl
%      \indent $|$ \n{ex}\gr{one optional space}
%              $|$ \gr{internal integer}\nl
%      \indent $|$ \gr{internal dimen} $|$ \gr{internal glue}
%      \end{disp}
%\end{itemize}
A dimension denotation is a somewhat complicated entity:
\begin{itemize} \item a \gr{factor} is an integer denotation,
 a decimal constant denotation (a number with an integral and
 a fractional part),
 or an \gr{internal integer}
 \begin{disp}\PopIndentLevel
      \gr{factor} $\longrightarrow$ \gr{normal integer} 
      $|$ \gr{decimal constant}\nl
      \gr{normal integer} $\longrightarrow$ \gr{integer denotation}\nl
      \indent $|$ \gr{internal integer}\nl
      \gr{decimal constant} $\longrightarrow$ \n{.$_{12}$}
      $|$ \n{,$_{12}$}\nl
      \indent $|$ \gr{digit}\gr{decimal constant}\nl
      \indent $|$ \gr{decimal constant}\gr{digit}
 \end{disp}
 An internal integer is a parameter that is `really' an
\alt
 integer (for instance, \cs{count0}), and not coerced from a dimension or glue.
 See Chapter~\ref{number}
 for the definition of various kinds of integers.
\item a \gr{unit of measure} can be 
 a \gr{physical unit}, that is, an ordinary unit such as~\n{cm} 
 (possibly preceded by \n{true}),
 an internal unit such as~\n{em}, but also an \gr{internal integer}
 (by conversion to scaled points),
 an \gr{internal dimen}, or an \gr{internal glue}.
 \begin{disp}\PopIndentLevel
      \gr{unit of measure} $\longrightarrow$
      \gr{optional spaces}\gr{internal unit}\nl
      \indent $|$ 
      \gr{optional \n{true}}\gr{physical unit}\gr{one optional space}\nl 
      \gr{internal unit} $\longrightarrow$ 
      \n{em}\gr{one optional space}\nl
      \indent $|$ \n{ex}\gr{one optional space}
              $|$ \gr{internal integer}\nl
      \indent $|$ \gr{internal dimen} $|$ \gr{internal glue}
      \end{disp}
\end{itemize}

%Some \gr{dimen}s are called \gr{special dimen}s:\label{special:dimen:list}
%\begin{disp}\gr{special dimen} $\longrightarrow$ \cs{prevdepth}\nl
%     \indent $|$ \cs{pagegoal} $|$ \cs{pagetotal} $|$ \cs{pagestretch}\nl
%     \indent $|$ \cs{pagefilstretch} $|$ \cs{pagefillstretch}\nl
%     \indent $|$ \cs{pagefilllstretch} $|$ \cs{pageshrink} $|$ \cs{pagedepth}
%     \end{disp}
%An assignment to any of these is
%called an \gr{intimate assignment}, and it is automatically
%global (see Chapter~\ref{group}). The meaning of these 
%dimensions is explained in Chapter \ref{page:break}, with the
%exception of \cs{prevdepth} which is treated in
%Chapter~\ref{baseline}.
Some \gr{dimen}s are called \gr{special dimen}s:\label{special:dimen:list}
\begin{disp}\gr{special dimen} $\longrightarrow$ \cs{prevdepth}\nl
     \indent $|$ \cs{pagegoal} $|$ \cs{pagetotal} $|$ \cs{pagestretch}\nl
     \indent $|$ \cs{pagefilstretch} $|$ \cs{pagefillstretch}\nl
     \indent $|$ \cs{pagefilllstretch} $|$ \cs{pageshrink} $|$ \cs{pagedepth}
     \end{disp}
An assignment to any of these is
called an \gr{intimate assignment}, and it is automatically
global (see Chapter~\ref{group}). The meaning of these 
dimensions is explained in Chapter \ref{page:break}, with the
exception of \cs{prevdepth} which is treated in
Chapter~\ref{baseline}.

%%\spoint Definition of glue
%\subsection{Definition of glue}
%\spoint Definition of glue
\subsection{Definition of glue}

%A \gr{glue} is either some form of glue variable, or
%a glue denotation with explicitly indicated stretch and
%shrink. Specifically,
%\begin{disp}\gr{glue} $\longrightarrow$ \gr{optional signs}\gr{internal glue}
%     $|$ \gr{dimen}\gr{stretch}\gr{shrink}\nl
%     \gr{internal glue} $\longrightarrow$ \gr{glue parameter}
%     $|$ \cs{lastskip}\nl 
%     \indent $|$ \gr{skipdef token} $|$ \cs{skip}\gr{8-bit number}\nl
%     \gr{glue parameter} $\longrightarrow$ \cs{abovedisplayshortskip}\nl
%     \indent $|$ \cs{abovedisplayskip} $|$ \cs{baselineskip}\nl
%     \indent $|$ \cs{belowdisplayshortskip} $|$ \cs{belowdisplayskip}\nl
%     \indent $|$ \cs{leftskip} $|$ \cs{lineskip} $|$ \cs{parfillskip}
%             $|$ \cs{parskip}\nl
%     \indent $|$ \cs{rightskip} $|$ \cs{spaceskip}
%             $|$ \cs{splittopskip} $|$ \cs{tabskip}\nl
%     \indent $|$ \cs{topskip} $|$ \cs{xspaceskip}
%\end{disp}
%The stretch and shrink components in a glue denotation
%are optional, but when both are specified they have to
%be given in sequence; they are defined as
%\begin{disp}
%\gr{stretch} $\longrightarrow$ \n{plus} \gr{dimen}
%      $|$ \n{plus}\gr{fil dimen} $|$ \gr{optional spaces}\nl
%\gr{shrink} $\longrightarrow$ \n{minus} \gr{dimen}
%      $|$ \n{minus}\gr{fil dimen} $|$ \gr{optional spaces}\nl
%\gr{fil dimen} $\longrightarrow$ \gr{optional signs}\gr{factor}%
%     \gr{fil unit}\gr{optional spaces}\nl
%\gr{fil unit} $\longrightarrow$ \n{ $|$ fil $|$ fill $|$ filll}
%\end{disp}
A \gr{glue} is either some form of glue variable, or
a glue denotation with explicitly indicated stretch and
shrink. Specifically,
\begin{disp}\gr{glue} $\longrightarrow$ \gr{optional signs}\gr{internal glue}
     $|$ \gr{dimen}\gr{stretch}\gr{shrink}\nl
     \gr{internal glue} $\longrightarrow$ \gr{glue parameter}
     $|$ \cs{lastskip}\nl 
     \indent $|$ \gr{skipdef token} $|$ \cs{skip}\gr{8-bit number}\nl
     \gr{glue parameter} $\longrightarrow$ \cs{abovedisplayshortskip}\nl
     \indent $|$ \cs{abovedisplayskip} $|$ \cs{baselineskip}\nl
     \indent $|$ \cs{belowdisplayshortskip} $|$ \cs{belowdisplayskip}\nl
     \indent $|$ \cs{leftskip} $|$ \cs{lineskip} $|$ \cs{parfillskip}
             $|$ \cs{parskip}\nl
     \indent $|$ \cs{rightskip} $|$ \cs{spaceskip}
             $|$ \cs{splittopskip} $|$ \cs{tabskip}\nl
     \indent $|$ \cs{topskip} $|$ \cs{xspaceskip}
\end{disp}
The stretch and shrink components in a glue denotation
are optional, but when both are specified they have to
be given in sequence; they are defined as
\begin{disp}
\gr{stretch} $\longrightarrow$ \n{plus} \gr{dimen}
      $|$ \n{plus}\gr{fil dimen} $|$ \gr{optional spaces}\nl
\gr{shrink} $\longrightarrow$ \n{minus} \gr{dimen}
      $|$ \n{minus}\gr{fil dimen} $|$ \gr{optional spaces}\nl
\gr{fil dimen} $\longrightarrow$ \gr{optional signs}\gr{factor}%
     \gr{fil unit}\gr{optional spaces}\nl
\gr{fil unit} $\longrightarrow$ \n{ $|$ fil $|$ fill $|$ filll}
\end{disp}

%The actual definition of \gr{fil unit} is recursive
%(see Chapter~\ref{gramm}), but these are the only valid
%possibilities.
The actual definition of \gr{fil unit} is recursive
(see Chapter~\ref{gramm}), but these are the only valid
possibilities.

%%\spoint Conversion of \gr{glue} to \gr{dimen}
%\subsection{Conversion of \gr{glue} to \gr{dimen}}
%\spoint Conversion of \gr{glue} to \gr{dimen}
\subsection{Conversion of \gr{glue} to \gr{dimen}}

%The grammar rule
%\begin{disp}\gr{dimen} $\longrightarrow$
%     \gr{factor}\gr{unit of measure}
%\end{disp}
%has some noteworthy consequences, caused by the fact
%that a \gr{unit of measure} need not look like a `unit of measure'
%at all (see the list above).
The grammar rule
\begin{disp}\gr{dimen} $\longrightarrow$
     \gr{factor}\gr{unit of measure}
\end{disp}
has some noteworthy consequences, caused by the fact
that a \gr{unit of measure} need not look like a `unit of measure'
at all (see the list above).

%For instance, from this definition we conclude that the statement
%\begin{verbatim}
%\dimen0=\lastpenalty\lastpenalty
%\end{verbatim}
%is
%syntactically correct because \cs{lastpenalty} can function
%both as an integer and as \gr{unit of measure} by taking
%its value in scaled points.
%After \verb>\penalty8> the \cs{dimen0} thus defined will
%have a size of~\n{64sp}.
For instance, from this definition we conclude that the statement
\begin{verbatim}
\dimen0=\lastpenalty\lastpenalty
\end{verbatim}
is
syntactically correct because \cs{lastpenalty} can function
both as an integer and as \gr{unit of measure} by taking
its value in scaled points.
After \verb>\penalty8> the \cs{dimen0} thus defined will
have a size of~\n{64sp}.

%More importantly, consider the case where the \gr{unit of measure} is
%an \gr{internal glue}, that is, any sort of glue parameter.
%Prefixing such a glue with a number (the \gr{factor})
%makes it a valid \gr{dimen} specification.
%Thus
%\begin{verbatim}
%\skip0=\skip1
%\end{verbatim}
%is very different
%from
%\begin{verbatim}
%\skip0=1\skip1
%\end{verbatim}
%The first statement makes
%\cs{skip0} equal to \cs{skip1}, the second converts
%the \cs{skip1} to a \gr{dimen} before assigning it.
%In other words, the \cs{skip0} defined by the second statement
%has no stretch or shrink.
More importantly, consider the case where the \gr{unit of measure} is
an \gr{internal glue}, that is, any sort of glue parameter.
Prefixing such a glue with a number (the \gr{factor})
makes it a valid \gr{dimen} specification.
Thus
\begin{verbatim}
\skip0=\skip1
\end{verbatim}
is very different
from
\begin{verbatim}
\skip0=1\skip1
\end{verbatim}
The first statement makes
\cs{skip0} equal to \cs{skip1}, the second converts
the \cs{skip1} to a \gr{dimen} before assigning it.
In other words, the \cs{skip0} defined by the second statement
has no stretch or shrink.


%%\spoint Registers for \cs{dimen} and \cs{skip}
%\subsection{Registers for \cs{dimen} and \cs{skip}}
%\spoint Registers for \cs{dimen} and \cs{skip}
\subsection{Registers for \cs{dimen} and \cs{skip}}

%\TeX\ has registers for storing \gr{dimen} and \gr{glue}
%values: the \csidx{dimen} and \csidx{skip} registers
%respectively. These are accessible by the expressions
%\begin{disp}\cs{dimen}\gr{number}\end{disp} and
%\begin{disp}\cs{skip}\gr{number}\end{disp}
%As with all registers of \TeX, these registers are
%numbered~0--255.
\TeX\ has registers for storing \gr{dimen} and \gr{glue}
values: the \csidx{dimen} and \csidx{skip} registers
respectively. These are accessible by the expressions
\begin{disp}\cs{dimen}\gr{number}\end{disp} and
\begin{disp}\cs{skip}\gr{number}\end{disp}
As with all registers of \TeX, these registers are
numbered~0--255.

%Synonyms for registers can be made with the \csidx{dimendef} and
%\csidx{skipdef} commands. Their syntax is
%\begin{Disp}\cs{dimendef}\gr{control sequence}\gr{equals}\gr{8-bit number}
%\end{Disp}
%and 
%\begin{Disp}\cs{skipdef}\gr{control sequence}\gr{equals}\gr{8-bit number}\end{Disp}
%For example, after \verb-\skipdef\foo=13- using \cs{foo}
%is equivalent to using \cs{skip13}.
Synonyms for registers can be made with the \csidx{dimendef} and
\csidx{skipdef} commands. Their syntax is
\begin{Disp}\cs{dimendef}\gr{control sequence}\gr{equals}\gr{8-bit number}
\end{Disp}
and 
\begin{Disp}\cs{skipdef}\gr{control sequence}\gr{equals}\gr{8-bit number}\end{Disp}
For example, after \verb-\skipdef\foo=13- using \cs{foo}
is equivalent to using \cs{skip13}.

%Macros \csidx{newdimen} and \csidx{newskip} exist in plain \TeX
%for allocating an unused dimen or skip register.
%These macros are defined to be \cs{outer} in the plain format.
Macros \csidx{newdimen} and \csidx{newskip} exist in plain \TeX
for allocating an unused dimen or skip register.
These macros are defined to be \cs{outer} in the plain format.

%\subsection{Arithmetic: addition}
\subsection{Arithmetic: addition}

%As for integer variables, arithmetic operations exist for
%\cstoidx advance\par
%\indextermbus{arithmetic on}{glue}\index{arithmetic! on glue|see{glue, arithmetic on}}:
%dimen, glue, and muglue (mathematical glue; see page~\pageref{muglue})
%variables.
As for integer variables, arithmetic operations exist for
\cstoidx advance\par
\indextermbus{arithmetic on}{glue}\index{arithmetic! on glue|see{glue, arithmetic on}}:
dimen, glue, and muglue (mathematical glue; see page~\pageref{muglue})
variables.

%The expressions
%\begin{Disp}\cs{advance}\gr{dimen variable}\gr{optional \n{by}}%
%     \gr{dimen}\nl
%     \cs{advance}\gr{glue variable}\gr{optional \n{by}}%
%     \gr{glue}\nl
%     \cs{advance}\gr{muglue variable}\gr{optional \n{by}}%
%     \gr{muglue}\end{Disp}
%add to the size of a dimen, glue, or muglue.
The expressions
\begin{Disp}\cs{advance}\gr{dimen variable}\gr{optional \n{by}}%
     \gr{dimen}\nl
     \cs{advance}\gr{glue variable}\gr{optional \n{by}}%
     \gr{glue}\nl
     \cs{advance}\gr{muglue variable}\gr{optional \n{by}}%
     \gr{muglue}\end{Disp}
add to the size of a dimen, glue, or muglue.

%Advancing a \gr{glue variable} by \gr{glue} is done by
%adding the natural sizes, and the stretch and shrink components.
%Because \TeX\ converts between \gr{glue} and \gr{dimen},
%it is possible to write for instance
%\begin{verbatim}
%\advance\skip1 by \dimen1
%\end{verbatim}
%or
%\begin{verbatim}
%\advance\dimen1 by \skip1
%\end{verbatim}
%In the first case  \cs{dimen1} is coerced to \gr{glue} without
%stretch or shrink; in the second case the \cs{skip1} is coerced
%to a \gr{dimen} by taking its natural size.
Advancing a \gr{glue variable} by \gr{glue} is done by
adding the natural sizes, and the stretch and shrink components.
Because \TeX\ converts between \gr{glue} and \gr{dimen},
it is possible to write for instance
\begin{verbatim}
\advance\skip1 by \dimen1
\end{verbatim}
or
\begin{verbatim}
\advance\dimen1 by \skip1
\end{verbatim}
In the first case  \cs{dimen1} is coerced to \gr{glue} without
stretch or shrink; in the second case the \cs{skip1} is coerced
to a \gr{dimen} by taking its natural size.

%%\spoint Arithmetic: multiplication and division
%\subsection{Arithmetic: multiplication and division}
%\spoint Arithmetic: multiplication and division
\subsection{Arithmetic: multiplication and division}

%Multiplication and division operations exist for glue
%\cstoidx multiply\par\cstoidx divide\par
%and dimensions. One may for instance write
%\begin{verbatim}
%\multiply\skip1 by 2
%\end{verbatim}
%which multiplies the natural size, and the stretch and shrink
%components of \cs{skip1} by~2.
Multiplication and division operations exist for glue
\cstoidx multiply\par\cstoidx divide\par
and dimensions. One may for instance write
\begin{verbatim}
\multiply\skip1 by 2
\end{verbatim}
which multiplies the natural size, and the stretch and shrink
components of \cs{skip1} by~2.

%The second operand of a \cs{multiply} or \cs{divide}
%operation can only be a \gr{number}, that is, an integer.
%Introducing the notion of \gr{numeric variable}:
%\begin{disp}\gr{numeric variable} $\longrightarrow$ \gr{integer variable}
%     $|$ \gr{dimen variable} \nl
%     \indent $|$ \gr{glue variable} $|$ \gr{muglue variable}\end{disp}
%these operations take the form
%\begin{Disp}\cs{multiply}\gr{numeric variable}\gr{optional \n{by}}%
%\gr{number}\end{Disp} 
%and
%\begin{Disp}\cs{divide}\gr{numeric variable}\gr{optional \n{by}}%
%\gr{number}\end{Disp}
The second operand of a \cs{multiply} or \cs{divide}
operation can only be a \gr{number}, that is, an integer.
Introducing the notion of \gr{numeric variable}:
\begin{disp}\gr{numeric variable} $\longrightarrow$ \gr{integer variable}
     $|$ \gr{dimen variable} \nl
     \indent $|$ \gr{glue variable} $|$ \gr{muglue variable}\end{disp}
these operations take the form
\begin{Disp}\cs{multiply}\gr{numeric variable}\gr{optional \n{by}}%
\gr{number}\end{Disp} 
and
\begin{Disp}\cs{divide}\gr{numeric variable}\gr{optional \n{by}}%
\gr{number}\end{Disp}

%Glue and dimen can be multiplied by 
%non-integer quantities:
%\begin{verbatim}
%\skip1=2.5\skip2
%\dimen1=.78\dimen2
%\end{verbatim}
%However, in the first line the \cs{skip2} is first coerced
%to a \gr{dimen} value by omitting its stretch and shrink.
Glue and dimen can be multiplied by 
non-integer quantities:
\begin{verbatim}
\skip1=2.5\skip2
\dimen1=.78\dimen2
\end{verbatim}
However, in the first line the \cs{skip2} is first coerced
to a \gr{dimen} value by omitting its stretch and shrink.

%%\point More about dimensions
%\section{More about dimensions}
%\point More about dimensions
\section{More about dimensions}

%%\spoint Units of measurement
%\subsection{Units of measurement}
%\spoint Units of measurement
\subsection{Units of measurement}

%In \TeX\ dimensions can be indicated in the following
%\indexterm{units of measurement}:
%\begin{description} 
%\item [centimetre]
%    denoted \n{cm} or 
%\item [millimetre]
%	denoted \n{mm}; these are SI~units ({\italic Syst\`eme International
%	d'Unit\'es}, the
%	international system of standard units of measurements).
%\item [inch]
%\n{in}; more common in the Anglo-American world.
%One inch is 2.54~centimetres.
%\item [pica]
%    denoted \n{pc}; one pica is 12~points.
%\item [point]
%    denoted \n{pt}; the common system
%for Anglo-American printers. One inch is 72.27 points.
%\item [didot point]
%    denoted \n{dd}; the common system for continental European printers.
%    Furthermore, 1157 didot points are 1238~points.
%\item [cicero]
%    denoted \n{cc}; one cicero is 12~didot points.
%\item [big point]
%    denoted \n{bp}; one inch is 72 big points.
%\item [scaled point]
%    denoted \n{sp}; this is the smallest unit in \TeX, and all measurements
%    are integral multiples of one scaled point.
%    There are $65\,536$ scaled points in a~point.
%\end{description}
In \TeX\ dimensions can be indicated in the following
\indexterm{units of measurement}:
\begin{description} 
\item [centimetre]
    denoted \n{cm} or 
\item [millimetre]
	denoted \n{mm}; these are SI~units ({\italic Syst\`eme International
	d'Unit\'es}, the
	international system of standard units of measurements).
\item [inch]
\n{in}; more common in the Anglo-American world.
One inch is 2.54~centimetres.
\item [pica]
    denoted \n{pc}; one pica is 12~points.
\item [point]
    denoted \n{pt}; the common system
for Anglo-American printers. One inch is 72.27 points.
\item [didot point]
    denoted \n{dd}; the common system for continental European printers.
    Furthermore, 1157 didot points are 1238~points.
\item [cicero]
    denoted \n{cc}; one cicero is 12~didot points.
\item [big point]
    denoted \n{bp}; one inch is 72 big points.
\item [scaled point]
    denoted \n{sp}; this is the smallest unit in \TeX, and all measurements
    are integral multiples of one scaled point.
    There are $65\,536$ scaled points in a~point.
\end{description}

%Decimal fractions can be written using both the
%Anglo-American system with the decimal point
%(for example, \n{1in}=\n{72.27pt})
%and the continental European system with a decimal
%comma; \n{1in}=\n{72,27pt}.
Decimal fractions can be written using both the
Anglo-American system with the decimal point
(for example, \n{1in}=\n{72.27pt})
and the continental European system with a decimal
comma; \n{1in}=\n{72,27pt}.

%Internally \TeX\ works with multiples of a smallest 
%dimension: the  scaled point.
%Dimensions larger (in absolute value) than $2^{30}-1$\n{sp},
%which is about 5.75~metres or 18.9~feet, are illegal.
Internally \TeX\ works with multiples of a smallest 
dimension: the  scaled point.
Dimensions larger (in absolute value) than $2^{30}-1$\n{sp},
which is about 5.75~metres or 18.9~feet, are illegal.

%Both the pica system and the didot system are of French
%origin: in 1737 the type founder Pierre Simon Fournier
%introduced typographical points based on the French foot.
%Although at first he introduced a system based on lines and
%points, he later took the point as unit:
%there are 72 points in an inch,
%which is one-twelfth of a foot. 
%About 1770 another founder, Fran\c{c}ois Ambroise Didot, introduced
%points based on the more common, and slightly longer,
%`pied du roi'.
Both the pica system and the didot system are of French
origin: in 1737 the type founder Pierre Simon Fournier
introduced typographical points based on the French foot.
Although at first he introduced a system based on lines and
points, he later took the point as unit:
there are 72 points in an inch,
which is one-twelfth of a foot. 
About 1770 another founder, Fran\c{c}ois Ambroise Didot, introduced
points based on the more common, and slightly longer,
`pied du roi'.

%%\spoint Dimension testing
%\subsection{Dimension testing}
%\spoint Dimension testing
\subsection{Dimension testing}

%Dimensions and natural sizes of glue can be compared with
%the \cs{ifdim} test. This takes the form
%\begin{disp}\cs{ifdim}\gr{dimen$_1$}\gr{relation}\gr{dimen$_2$}\end{disp}
%where the relation can be an \n>, \n<, or~\n= token, 
%all of category~12\index{category!12}.
Dimensions and natural sizes of glue can be compared with
the \cs{ifdim} test. This takes the form
\begin{disp}\cs{ifdim}\gr{dimen$_1$}\gr{relation}\gr{dimen$_2$}\end{disp}
where the relation can be an \n>, \n<, or~\n= token, 
all of category~12\index{category!12}.

%%\spoint Defined dimensions
%\subsection{Defined dimensions}
%\spoint Defined dimensions
\subsection{Defined dimensions}

%\begin{inventory}
%\item [\cs{z@}]
% \n{0pt}
\begin{inventory}
\item [\cs{z@}]
 \n{0pt}

%\item [\cs{maxdimen}] 
%      \n{16383.99999pt}; the largest legal dimension.
%\end{inventory}
\item [\cs{maxdimen}] 
      \n{16383.99999pt}; the largest legal dimension.
\end{inventory}

%These \gr{dimen}s are predefined in the plain format;
%for instance
%\begin{verbatim}
%\newdimen\z@ \z@=0pt
%\end{verbatim}
%Using such abbreviations for commonly used dimensions
%has at least two advantages. First of all it saves main memory
%if such a dimension occurs in a macro: a control sequence
%is one token, whereas a string such as \n{0pt} takes three.
%Secondly, it saves time in processing, as \TeX\ does not need
%to perform conversions to arrive at the correct type of
%object.
These \gr{dimen}s are predefined in the plain format;
for instance
\begin{verbatim}
\newdimen\z@ \z@=0pt
\end{verbatim}
Using such abbreviations for commonly used dimensions
has at least two advantages. First of all it saves main memory
if such a dimension occurs in a macro: a control sequence
is one token, whereas a string such as \n{0pt} takes three.
Secondly, it saves time in processing, as \TeX\ does not need
to perform conversions to arrive at the correct type of
object.

%Control sequences such as \cs{z@}
%are only available to a user who changes the
%category code of the `at' sign. Ordinarily, these control sequences
%appear only in the macros defined in packages such as the
%plain format.
Control sequences such as \cs{z@}
are only available to a user who changes the
category code of the `at' sign. Ordinarily, these control sequences
appear only in the macros defined in packages such as the
plain format.

%%\point More about glue
%\section{More about glue}
%\point More about glue
\section{More about glue}

%Glue items can be added to a vertical list with one of the
%\alt
%commands \csidx{vskip}\gr{glue}, \cs{vfil}, \cs{vfill}, \cs{vss} or
%\cs{vfilneg}; 
%glue items can be added to a horizontal list with one of the
%commands \csidx{hskip}\gr{glue}, \cs{hfil}, \cs{hfill}, \cs{hss} or
%\cs{hfilneg}. We will now treat the properties of glue.
Glue items can be added to a vertical list with one of the
\alt
commands \csidx{vskip}\gr{glue}, \cs{vfil}, \cs{vfill}, \cs{vss} or
\cs{vfilneg}; 
glue items can be added to a horizontal list with one of the
commands \csidx{hskip}\gr{glue}, \cs{hfil}, \cs{hfill}, \cs{hss} or
\cs{hfilneg}. We will now treat the properties of glue.

%%\spoint Stretch and shrink
%\subsection{Stretch and shrink}
%\spoint Stretch and shrink
\subsection{Stretch and shrink}

%In the syntax given above, \gr{glue} was defined as having
%\begin{itemize}
%\item a `natural size', which is a \gr{dimen}, and optionally
%\item a \indexterm{stretch} and \indexterm{shrink}
%  component\indexterm{stretch}\indexterm{shrink}
%  \index{glue!stretch component of}\index{glue!shrink
%    component of} built out of a \gr{fil dimen}.
%\end{itemize}
In the syntax given above, \gr{glue} was defined as having
\begin{itemize}
\item a `natural size', which is a \gr{dimen}, and optionally
\item a \indexterm{stretch} and \indexterm{shrink}
  component\indexterm{stretch}\indexterm{shrink}
  \index{glue!stretch component of}\index{glue!shrink
    component of} built out of a \gr{fil dimen}.
\end{itemize}

%Each list that \TeX\ builds has amounts of stretch and shrink
%(possibly zero),
%which are the sum of the
%stretch and shrink components of individual pieces of glue in the list. 
%Stretch and shrink are used if the context in which the list
%appears requires it to assume a size that is different from
%its natural size.
Each list that \TeX\ builds has amounts of stretch and shrink
(possibly zero),
which are the sum of the
stretch and shrink components of individual pieces of glue in the list. 
Stretch and shrink are used if the context in which the list
appears requires it to assume a size that is different from
its natural size.

%There is an important difference in behaviour between stretch
%and shrink components when they are finite \ldash that is,
%when the \gr{fildimen} is not \n{fil}(\n{l}(\n{l})). 
%A~finite amount of shrink is indeed the maximum shrink
%that \TeX\ will take: the amount of glue specified
%as
%\begin{verbatim}
%5pt minus 3pt
%\end{verbatim}
%can shrink to \n{2pt}, but not further.
%In contrast to this, a finite amount of stretch 
%can be stretched arbitrarily far. 
%Such arbitrary stretching
%has a large `badness', however.
%Badness calculation is treated below.
There is an important difference in behaviour between stretch
and shrink components when they are finite \ldash that is,
when the \gr{fildimen} is not \n{fil}(\n{l}(\n{l})). 
A~finite amount of shrink is indeed the maximum shrink
that \TeX\ will take: the amount of glue specified
as
\begin{verbatim}
5pt minus 3pt
\end{verbatim}
can shrink to \n{2pt}, but not further.
In contrast to this, a finite amount of stretch 
can be stretched arbitrarily far. 
Such arbitrary stretching
has a large `badness', however.
Badness calculation is treated below.

%\begin{example}
%The sequence with natural size \n{20pt}
%\begin{verbatim}
%\hskip 10pt plus 2pt \hskip 10pt plus 3pt
%\end{verbatim}
%has \n{5pt} of stretch, but it has no shrink. In
%\begin{verbatim}
%\hskip 10pt minus 2pt \hskip 10pt plus 3pt
%\end{verbatim}
%there is \n{3pt} of stretch, and \n{2pt} of shrink,
%so its minimal size is~\n{18pt}. 
\begin{example}
The sequence with natural size \n{20pt}
\begin{verbatim}
\hskip 10pt plus 2pt \hskip 10pt plus 3pt
\end{verbatim}
has \n{5pt} of stretch, but it has no shrink. In
\begin{verbatim}
\hskip 10pt minus 2pt \hskip 10pt plus 3pt
\end{verbatim}
there is \n{3pt} of stretch, and \n{2pt} of shrink,
so its minimal size is~\n{18pt}. 

%Positive shrink is not the same as negative stretch:
%\begin{verbatim}
%\hskip 10pt plus -2pt \hskip 10pt plus 3pt
%\end{verbatim}
%looks a lot like the previous example, but it cannot
%be shrunk as there are no \hbox{\n{minus}\gr{dimen}}
%specifications. It does have \n{1pt} of stretch, however.
Positive shrink is not the same as negative stretch:
\begin{verbatim}
\hskip 10pt plus -2pt \hskip 10pt plus 3pt
\end{verbatim}
looks a lot like the previous example, but it cannot
be shrunk as there are no \hbox{\n{minus}\gr{dimen}}
specifications. It does have \n{1pt} of stretch, however.

%This is another example of negative amounts of shrink and stretch.
%It is not possible to stretch
%glue (in the informal sense) by shrinking it (in the technical
%sense):
%\begin{verbatim}
%\hbox to 5cm{a\hskip 0cm minus -1fil}
%\end{verbatim}
%is an underfull box, because \TeX\ looks for a \n{plus}~\gr{dimen}
%specification when it needs to stretch the contents.
This is another example of negative amounts of shrink and stretch.
It is not possible to stretch
glue (in the informal sense) by shrinking it (in the technical
sense):
\begin{verbatim}
\hbox to 5cm{a\hskip 0cm minus -1fil}
\end{verbatim}
is an underfull box, because \TeX\ looks for a \n{plus}~\gr{dimen}
specification when it needs to stretch the contents.

%Finally,
%\begin{verbatim}
%\hskip 10pt plus -3pt \hskip 10pt plus 3pt
%\end{verbatim}
%can neither stretch nor shrink.
%The fact that there is only stretch
%available means that the sequence cannot
%shrink. However, the stretch components cancel out: the 
%total stretch is zero. Another way of looking at this
%is to consider that for each point that the second glue item would
%stretch, the first one would `stretch back' one point.
%\end{example}
Finally,
\begin{verbatim}
\hskip 10pt plus -3pt \hskip 10pt plus 3pt
\end{verbatim}
can neither stretch nor shrink.
The fact that there is only stretch
available means that the sequence cannot
shrink. However, the stretch components cancel out: the 
total stretch is zero. Another way of looking at this
is to consider that for each point that the second glue item would
stretch, the first one would `stretch back' one point.
\end{example}

%Any amount of infinite stretch or shrink overpowers all
%finite stretch or shrink available:
%\begin{verbatim}
%\hbox to 5cm{\hskip 0cm plus 16384pt 
%              text\hskip 0cm plus 0.0001fil}
%\end{verbatim}
%has the \n{text} at the extreme left of the box.
%There are three orders of `infinity', each  one infinitely
%stronger than the previous one:
%\begin{verbatim}
%\hbox to 5cm{\hskip 0cm plus 16384fil
%              text\hskip 0cm plus 0.0001fill}
%\end{verbatim}
%and
%\begin{verbatim}
%\hbox to 5cm{\hskip 0cm plus 16384fill
%              text\hskip 0cm plus 0.0001filll}
%\end{verbatim}
%both have the \n{text} at the left end of the box.
Any amount of infinite stretch or shrink overpowers all
finite stretch or shrink available:
\begin{verbatim}
\hbox to 5cm{\hskip 0cm plus 16384pt 
              text\hskip 0cm plus 0.0001fil}
\end{verbatim}
has the \n{text} at the extreme left of the box.
There are three orders of `infinity', each  one infinitely
stronger than the previous one:
\begin{verbatim}
\hbox to 5cm{\hskip 0cm plus 16384fil
              text\hskip 0cm plus 0.0001fill}
\end{verbatim}
and
\begin{verbatim}
\hbox to 5cm{\hskip 0cm plus 16384fill
              text\hskip 0cm plus 0.0001filll}
\end{verbatim}
both have the \n{text} at the left end of the box.



%%\spoint Glue setting
%\subsection{Glue setting}
%\spoint Glue setting
\subsection{Glue setting}

%In the process of \indextermsub{glue}{setting}, the desired width (or height)
%of a box is compared with the natural dimension of its contents,
%which is the sum of all natural dimensions of boxes and globs of glue.
%If the two differ, any available stretchability or shrinkability is used
%to bridge the gap.
%To attain the desired dimension of the box
%only the glue of the highest available order is set:
%each piece of glue of that order is stretched or shrunk by the
%same ratio.
In the process of \indextermsub{glue}{setting}, the desired width (or height)
of a box is compared with the natural dimension of its contents,
which is the sum of all natural dimensions of boxes and globs of glue.
If the two differ, any available stretchability or shrinkability is used
to bridge the gap.
To attain the desired dimension of the box
only the glue of the highest available order is set:
each piece of glue of that order is stretched or shrunk by the
same ratio.

%For example, in
%\begin{verbatim}
%\hbox to 6pt{\hskip 0pt plus 3pt \hskip 0pt plus 9pt}
%\end{verbatim}
%the natural size of the box is~\n{0pt}, and
%the total stretch is~\n{12pt}. In order to obtain a box
%of~\n{6pt} each glue item is set with a stretch ratio
%of~$1/2$. Thus the result is equivalent to
%\begin{verbatim}
%\hbox {\hskip 1.5pt \hskip 4.5pt}
%\end{verbatim}
%Only the highest order of stretch or shrink is used:
%in
%\begin{verbatim}
%\hbox to 6pt{\hskip 0pt plus 1fil \hskip 0pt plus 9pt}
%\end{verbatim}
%the second glue  will assume its natural size of~\n{0pt},
%and only the first   glue will be stretched.
For example, in
\begin{verbatim}
\hbox to 6pt{\hskip 0pt plus 3pt \hskip 0pt plus 9pt}
\end{verbatim}
the natural size of the box is~\n{0pt}, and
the total stretch is~\n{12pt}. In order to obtain a box
of~\n{6pt} each glue item is set with a stretch ratio
of~$1/2$. Thus the result is equivalent to
\begin{verbatim}
\hbox {\hskip 1.5pt \hskip 4.5pt}
\end{verbatim}
Only the highest order of stretch or shrink is used:
in
\begin{verbatim}
\hbox to 6pt{\hskip 0pt plus 1fil \hskip 0pt plus 9pt}
\end{verbatim}
the second glue  will assume its natural size of~\n{0pt},
and only the first   glue will be stretched.

%\TeX\ will never exceed the maximum value of a finite
%amount of shrink.
%A~box that cannot be shrunk enough is called `overfull'.
%Finite stretchability can be exceeded to provide an
%escape in difficult situations; however, \TeX\ is likely 
%to give an \verb-Underfull \hbox- message about this
%(see page~\pageref{over/underfull}).
%For an example of infinite shrink see page~\pageref{rlap}.
\TeX\ will never exceed the maximum value of a finite
amount of shrink.
A~box that cannot be shrunk enough is called `overfull'.
Finite stretchability can be exceeded to provide an
escape in difficult situations; however, \TeX\ is likely 
to give an \verb-Underfull \hbox- message about this
(see page~\pageref{over/underfull}).
For an example of infinite shrink see page~\pageref{rlap}.

%%\spoint Badness
%\subsection{Badness}
%\spoint Badness
\subsection{Badness}

%When stretching or shrinking a list \TeX\ calculates 
%\indexterm{badness}\index{badness! calculation}
%badness based on the
%ratio between actual stretch and the amount of stretch
%present in the line. See Chapter~\ref{line:break}
%for the application  of badness to the paragraph algorithm.
When stretching or shrinking a list \TeX\ calculates 
\indexterm{badness}\index{badness! calculation}
badness based on the
ratio between actual stretch and the amount of stretch
present in the line. See Chapter~\ref{line:break}
for the application  of badness to the paragraph algorithm.

%The formula for badness of a list that is stretched (shrunk) is
%\label{bad:form}\message{Check roman min}
%\begin{disp} $\displaystyle b=\hbox{min}\left(10\,000,
%100\times \left({\hbox{actual amount stretched (shrunk)}
%\over\hbox{possible amount of stretch (shrink)}}\right)^3\right)$\end{disp}
%In reality \TeX\ uses a slightly different formula that is
%easier to calculate, but behaves the same. Since glue setting is
%one of the main activities of \TeX, this must be performed
%as efficiently as possible.
The formula for badness of a list that is stretched (shrunk) is
\label{bad:form}\message{Check roman min}
\begin{disp} $\displaystyle b=\hbox{min}\left(10\,000,
100\times \left({\hbox{actual amount stretched (shrunk)}
\over\hbox{possible amount of stretch (shrink)}}\right)^3\right)$\end{disp}
In reality \TeX\ uses a slightly different formula that is
easier to calculate, but behaves the same. Since glue setting is
one of the main activities of \TeX, this must be performed
as efficiently as possible.

%This formula lets the badness be a reasonably small number
%if the glue set ratio (the fraction in the above expression)
%is reasonably small, but will let it grow rapidly once
%the ratio is more than~1. Badness is infinite if the
%glue would have to shrink more than the allotted amount;
%stretching glue beyond its maximum is possible, so this
%provides an  escape for very difficult lines of text or pages.
This formula lets the badness be a reasonably small number
if the glue set ratio (the fraction in the above expression)
is reasonably small, but will let it grow rapidly once
the ratio is more than~1. Badness is infinite if the
glue would have to shrink more than the allotted amount;
stretching glue beyond its maximum is possible, so this
provides an  escape for very difficult lines of text or pages.

%In \TeX3, the \cs{badness} parameter records the badness
%of the most recently formed box.
In \TeX3, the \cs{badness} parameter records the badness
of the most recently formed box.

%%\spoint Glue and breaking
%\subsection{Glue and breaking}
%\spoint Glue and breaking
\subsection{Glue and breaking}

%\TeX\ can break lines and pages in several kinds of places.
%One of these places is before a glue item. 
%The glue is then discarded. For line breaks this is treated
%in Chapter~\ref{line:break}, 
%for page breaks see Chapter~\ref{page:break}.
\TeX\ can break lines and pages in several kinds of places.
One of these places is before a glue item. 
The glue is then discarded. For line breaks this is treated
in Chapter~\ref{line:break}, 
for page breaks see Chapter~\ref{page:break}.

%There are two macros in plain \TeX, \csidx{hglue} and \csidx{vglue},
%that give non-disappearing glue in horizontal and
%vertical mode respectively. For the horizontal case this is
%accomplished by
%placing:
%\begin{verbatim}
%\vrule width 0pt \nobreak \hskip ...
%\end{verbatim}
%Because \TeX\ breaks at the front end of glue,
%this glue will always stay attached to the rule,
%and will therefore never disappear.
%The actual macro definitions are somewhat more complicated,
%because they take care to preserve the \cs{spacefactor} and the
%\cs{prevdepth}.
There are two macros in plain \TeX, \csidx{hglue} and \csidx{vglue},
that give non-disappearing glue in horizontal and
vertical mode respectively. For the horizontal case this is
accomplished by
placing:
\begin{verbatim}
\vrule width 0pt \nobreak \hskip ...
\end{verbatim}
Because \TeX\ breaks at the front end of glue,
this glue will always stay attached to the rule,
and will therefore never disappear.
The actual macro definitions are somewhat more complicated,
because they take care to preserve the \cs{spacefactor} and the
\cs{prevdepth}.

%%\spoint \cs{kern}
%\subsection{\cs{kern}}
%\spoint \cs{kern}
\subsection{\cs{kern}}

%The \csidx{kern} command specifies
%a~kern item in whatever mode \TeX\ is currently
%in. A~kern item is much like a glue item without
%stretch or shrink.
%It differs from glue in that it is
%in general not a legal breakpoint. Thus in
%\begin{verbatim}
%.. text .. \hbox{a}\kern0pt\hbox{b}
%\end{verbatim}
%\TeX\ will not break lines in between the boxes; in
%\begin{verbatim}
%.. text .. \hbox{a}\hskip0pt\hbox{b}
%\end{verbatim}
%a line can be broken in between the boxes.
The \csidx{kern} command specifies
a~kern item in whatever mode \TeX\ is currently
in. A~kern item is much like a glue item without
stretch or shrink.
It differs from glue in that it is
in general not a legal breakpoint. Thus in
\begin{verbatim}
.. text .. \hbox{a}\kern0pt\hbox{b}
\end{verbatim}
\TeX\ will not break lines in between the boxes; in
\begin{verbatim}
.. text .. \hbox{a}\hskip0pt\hbox{b}
\end{verbatim}
a line can be broken in between the boxes.

%However, if a kern is followed by glue, \TeX\ can break at the
%kern (provided that it is not in math mode). 
%In horizontal mode
%both the kern and the glue then disappear in the break.
%In vertical mode they are discarded when they are moved to
%the (empty) current page after the material before
%the break has been disposed of by the output routine 
%(see Chapter~\ref{page:break}).
However, if a kern is followed by glue, \TeX\ can break at the
kern (provided that it is not in math mode). 
In horizontal mode
both the kern and the glue then disappear in the break.
In vertical mode they are discarded when they are moved to
the (empty) current page after the material before
the break has been disposed of by the output routine 
(see Chapter~\ref{page:break}).

%%\spoint Glue and modes
%\subsection{Glue and modes}
%\spoint Glue and modes
\subsection{Glue and modes}

%All horizontal skip commands are \gr{horizontal command}s and
%all vertical skip commands are \gr{vertical commands}s.
%This means that, for instance, an \cs{hskip} command
%makes \TeX\ start a paragraph if it is given in vertical mode.
%The \cs{kern} command can be given in both modes.
All horizontal skip commands are \gr{horizontal command}s and
all vertical skip commands are \gr{vertical commands}s.
This means that, for instance, an \cs{hskip} command
makes \TeX\ start a paragraph if it is given in vertical mode.
The \cs{kern} command can be given in both modes.

%%\spoint The last  glue item in a list: backspacing
%\subsection{The last  glue item in a list: backspacing}
%\spoint The last  glue item in a list: backspacing
\subsection{The last  glue item in a list: backspacing}

%The last glue item in a list can be measured, and
%it can be removed in all modes but external vertical mode.
%The internal variables
%\csidx{lastskip} and  \csidx{lastkern} can be used
%to measure the last glob of glue in all modes;
%if the last glue was not a skip or kern respectively
%they give~\n{0pt}.
%In math mode the \cs{lastskip}
%functions as \gr{internal muglue}, but in general
%it classifies as \gr{internal glue}.
%The \cs{lastskip} and \cs{lastkern}
%are also \n{0pt} if that was the size of the last glue or
%kern item on the list.
The last glue item in a list can be measured, and
it can be removed in all modes but external vertical mode.
The internal variables
\csidx{lastskip} and  \csidx{lastkern} can be used
to measure the last glob of glue in all modes;
if the last glue was not a skip or kern respectively
they give~\n{0pt}.
In math mode the \cs{lastskip}
functions as \gr{internal muglue}, but in general
it classifies as \gr{internal glue}.
The \cs{lastskip} and \cs{lastkern}
are also \n{0pt} if that was the size of the last glue or
kern item on the list.

%The operations\label{unskip}
%\csidx{unskip} and \csidx{unkern} remove the last item of a list,
%if this is a glue or kern respectively. They have no effect
%in external vertical mode; in that case the
%best substitute is 
%\verb=\vskip-\lastskip= 
%and~\verb=\kern-\lastkern=.
The operations\label{unskip}
\csidx{unskip} and \csidx{unkern} remove the last item of a list,
if this is a glue or kern respectively. They have no effect
in external vertical mode; in that case the
best substitute is 
\verb=\vskip-\lastskip= 
and~\verb=\kern-\lastkern=.

%In the process of paragraph building \TeX\ itself performs
%an important \cs{unskip}: a~paragraph ending with a
%white line will have a space token inserted by \TeX's input processor.
%This is removed by an \cs{unskip} before the \cs{parfillskip} glue
%(see Chapter~\ref{par:end}) is inserted.
In the process of paragraph building \TeX\ itself performs
an important \cs{unskip}: a~paragraph ending with a
white line will have a space token inserted by \TeX's input processor.
This is removed by an \cs{unskip} before the \cs{parfillskip} glue
(see Chapter~\ref{par:end}) is inserted.

%Glue is treated by \TeX\ as a special case of leaders,
%which becomes apparent when \cs{unskip} is applied to
%leaders: they are removed.
Glue is treated by \TeX\ as a special case of leaders,
which becomes apparent when \cs{unskip} is applied to
leaders: they are removed.

%%\spoint Examples of backspacing
%\subsection{Examples of backspacing}
%\spoint Examples of backspacing
\subsection{Examples of backspacing}

%The plain \TeX\ macro \csidx{removelastskip} is defined
%as
%\begin{verbatim}
%\ifdim\lastskip=0pt \else \vskip-\lastskip \fi
%\end{verbatim}
%If the last item on the list was a glue, this macro will
%backspace by its value, provided its natural size was not zero.
%In all other cases, nothing is added to the list.
The plain \TeX\ macro \csidx{removelastskip} is defined
as
\begin{verbatim}
\ifdim\lastskip=0pt \else \vskip-\lastskip \fi
\end{verbatim}
If the last item on the list was a glue, this macro will
backspace by its value, provided its natural size was not zero.
In all other cases, nothing is added to the list.

%Sometimes an intelligent version of commands such as \cs{vskip}
%is necessary, in the sense that two subsequent skip commands
%should result only in the larger of the two glue amounts.
%On page~\pageref{skip:scheme} such a macro is used:
%\begin{verbatim}
%\newskip\tempskipa
%\def\vspace#1{\tempskipa=#1\relax
%    \ifvmode \ifdim\tempskipa<\lastskip 
%             \else \vskip-\lastskip \vskip\tempskipa
%             \fi
%    \else \vskip\tempskipa \fi}
%\end{verbatim}
%First of all, this tests whether the mode is vertical; 
%if not, the argument can safely be placed.
%Copying the argument into a skip register is necessary
%because \cs{v\-space}\verb>{2pt plus 3pt}> would lead to 
%problems in an \verb>\ifdim#1<\lastskip> test.
Sometimes an intelligent version of commands such as \cs{vskip}
is necessary, in the sense that two subsequent skip commands
should result only in the larger of the two glue amounts.
On page~\pageref{skip:scheme} such a macro is used:
\begin{verbatim}
\newskip\tempskipa
\def\vspace#1{\tempskipa=#1\relax
    \ifvmode \ifdim\tempskipa<\lastskip 
             \else \vskip-\lastskip \vskip\tempskipa
             \fi
    \else \vskip\tempskipa \fi}
\end{verbatim}
First of all, this tests whether the mode is vertical; 
if not, the argument can safely be placed.
Copying the argument into a skip register is necessary
because \cs{v\-space}\verb>{2pt plus 3pt}> would lead to 
problems in an \verb>\ifdim#1<\lastskip> test.

%If the surrounding mode was vertical, the argument
%should only be placed if it is not less than what is
%already there. The macro would be incorrect
%if the test  read
%\begin{verbatim}
%        \ifdim\tempskipa>\lastskip 
%            \vskip-\lastskip \vskip\tempskipa
%        \fi
%\end{verbatim}
%In this case the sequence
%\begin{verbatim}
%... last word.\par \vspace{0pt plus 1fil}
%\end{verbatim}
%would not place any glue, because after
%the \cs{par} we are in vertical mode and
%\cs{lastskip} has a value of \n{0pt}.
If the surrounding mode was vertical, the argument
should only be placed if it is not less than what is
already there. The macro would be incorrect
if the test  read
\begin{verbatim}
        \ifdim\tempskipa>\lastskip 
            \vskip-\lastskip \vskip\tempskipa
        \fi
\end{verbatim}
In this case the sequence
\begin{verbatim}
... last word.\par \vspace{0pt plus 1fil}
\end{verbatim}
would not place any glue, because after
the \cs{par} we are in vertical mode and
\cs{lastskip} has a value of \n{0pt}.

%%\spoint Glue in trace output
%\subsection{Glue in trace output}
%\spoint Glue in trace output
\subsection{Glue in trace output}

%If the workings of \TeX\ are traced by setting
%\cs{tracingoutput} positive, or if \TeX\ 
%writes a box to the log file 
%(because of a \cs{showbox} command, or because it
%is overfull or underfull),
%glue  is denoted by the control sequence \cs{glue}.
%This is not a \TeX\ command; it merely indicates the presence
%of glue in the current list.
If the workings of \TeX\ are traced by setting
\cs{tracingoutput} positive, or if \TeX\ 
writes a box to the log file 
(because of a \cs{showbox} command, or because it
is overfull or underfull),
glue  is denoted by the control sequence \cs{glue}.
This is not a \TeX\ command; it merely indicates the presence
of glue in the current list.

%The box representation that \TeX\ generated from,
%\alt
%for instance, \cs{showbox}
%inserts a space after every explicit \cs{kern},
%but no space is inserted after an implicit
%kern that was inserted by the kerning information in the font
%\n{tfm} file. Thus \hbox{\verb-\kern 2.0pt-} denotes a kern
%that was inserted by the user or by a macro, and
%\verb-\kern2.0pt- denotes an implicit kern.
The box representation that \TeX\ generated from,
\alt
for instance, \cs{showbox}
inserts a space after every explicit \cs{kern},
but no space is inserted after an implicit
kern that was inserted by the kerning information in the font
\n{tfm} file. Thus \hbox{\verb-\kern 2.0pt-} denotes a kern
that was inserted by the user or by a macro, and
\verb-\kern2.0pt- denotes an implicit kern.

%Glue that is inserted automatically (\cs{topskip}, \cs{baselineskip},
%et cetera) is denoted by name in \TeX's trace output.
%For example, the box
%\begin{verbatim}
%\vbox{\hbox{Vo}\hbox{b}}
%\end{verbatim}
%looks like
%\begin{verbatim}
%\vbox(18.83331+0.0)x11.66669
%.\hbox(6.83331+0.0)x11.66669
%..\tenrm V
%..\kern-0.83334
%..\tenrm o
%.\glue(\baselineskip) 5.05556
%.\hbox(6.94444+0.0)x5.55557
%..\tenrm b
%\end{verbatim}
%Note the implicit kern inserted between `V' and~`o'.
Glue that is inserted automatically (\cs{topskip}, \cs{baselineskip},
et cetera) is denoted by name in \TeX's trace output.
For example, the box
\begin{verbatim}
\vbox{\hbox{Vo}\hbox{b}}
\end{verbatim}
looks like
\begin{verbatim}
\vbox(18.83331+0.0)x11.66669
.\hbox(6.83331+0.0)x11.66669
..\tenrm V
..\kern-0.83334
..\tenrm o
.\glue(\baselineskip) 5.05556
.\hbox(6.94444+0.0)x5.55557
..\tenrm b
\end{verbatim}
Note the implicit kern inserted between `V' and~`o'.

%\endofchapter
%%%%% end of input file [glue]
\endofchapter
%%%% end of input file [glue]

\end{document}
